\documentclass[12pt]{report}
\usepackage{cite}
\usepackage{tikz}
\usepackage{verbatim}
\usepackage[font=small,labelfont=bf]{caption}
\usepackage{graphicx}
\usepackage{fullpage}
%\usepackage{soul}
\usepackage[german]{babel}
\usepackage{float}
\usepackage{mathtools}
\usepackage{amsfonts}
\usepackage{hyperref}
\usepackage{pdfpages}
%\usepackage{amssymb}
%\usepackage{paralist}
%\usepackage{fancyhdr}
\usepackage{amsthm}
\usepackage{setspace}
\usepackage{subfig}
\usepackage{listings}
\usepackage{xspace}
\usepackage{listings}
\usepackage{fontawesome}
\usepackage{amsmath}
\usepackage{amssymb}
\usepackage{xcolor}
\usepackage{soul}

\usepackage{titlesec}

\titleformat{\chapter}[display]
  {\normalfont\bfseries}{}{0pt}{\Huge}
\newtheoremstyle{nonitalic}
  {\topsep}   % Space above
  {\topsep}   % Space below
  {\normalfont}  % Body font
  {}          % Indent amount (empty = no indent, \parindent = para indent)
  {\bfseries} % Theorem head font
  {.}         % Punctuation after theorem head
  { }         % Space after theorem head (\newline = linebreak)
  {}          % Theorem head spec (can be left empty, meaning ‘normal’)
\newtheorem{theorem}{Satz}
\newtheorem{lemma}[theorem]{Lemma}
\newtheorem{Beweis}[theorem]{Beweis}
\newtheorem{Korollar}[theorem]{Korollar}
\newtheorem{Definition}{Definition}
\theoremstyle{nonitalic}
\newtheorem*{Notation}{Notation}
\newtheorem*{Beispiel}{Beispiel}
\newtheorem{Bemerkung}{Bemerkung}
\newtheorem*{Konstruktion*}{Konstruktion}
\newcommand{\PsubT}{\text{Principal subtree }}
\newcommand{\linpreft}{\text{linear preferential attachment trees }}



% Title Page
% LTeX: language=de-DE
\begin{document}
\allowdisplaybreaks
\author{Victor Velesco}
\include{chapters/Titelblatt}
\includepdf[pages=1]{Selbststandigkeitserklarung.pdf}
\tableofcontents
\chapter{Einführung}
% LTeX: language=de-DE
Die vorliegende Bachelorarbeit ist eine Ausarbeitung von der wissenschaftlichen Publikation \glqq Random Recursive Trees and Preferential Attachment Trees are Random Split Trees\grqq, von Svante Janson \cite{janson2019random}.\\
Die Publikation von Janson verallgemeinert die von Devroye \cite{devroye1998universal} eingeführten \textit{Split trees}, indem auch Bäume mit unendlichen Ausgangsgraden, also Anzahl von Kindern pro Wurzel, zugelassen werden. Präziser formuliert, werden sogenannte \textit{Linear Preferential Attachment Trees} in die Vorlage der Split-Bäume aufgenommen. Je nach Parameter Wahl umfassen die \textit{Preferential Trees} unter anderem die \textit{m-ary, standard preferential und zufälligen rekursiven} Bäume und Eigenschaften der jeweiligen Bäume werden in Kapiteln 3 und 4 näher betrachtet. Da die Publikation stark auf \textit{Kingman's theory of exchangeable partitions} \cite{kingman1978representation,kingman1982coalescent} als auch auf \textit{Pitman's chinese restaurant process} \cite{pitman2006combinatorial,pitman1995exchangeable} beruht, wird im ersten Kapitel ein Überblick über diese Themengebiete verschaffen. Anschließend wird das Haupttheorem zur Einbettung in Kapitel 2 bewiesen, gefolgt von einer Anwendung zu Tiefen in Kapitel 3. Im letzten Kapitel wird ein Spezialfall der \textit{Preferential Trees} gesondert behandelt.

\begin{section}{Definitionen}
% LTeX: language=de-DE
\begin{Definition}
   Sei $F$ eine endliche Menge. Dann ist eine Partition von $F$ eine ungeordnete Sammlung von nichtleeren, disjunkten Untermengen $\{A_1 ... A_k\}$, mit $\cup_{i=1}^{k}A_k = F$
\end{Definition}
\end{section}
\begin{section}{Kingman Paintbox representation}
% LTeX: language=de-DE
\begin{Definition}{Konsistenz}\\
    Wir nennen eine Folge von austauschbaren zufälligen Partitionen $(\Pi_n)$ \textit{konsistent}, falls die Restriktion $\Pi_{m,n}$ von $\Pi_m$ zu $[n]$ für alle $m > n$ die gleiche Verteilung hat wie $\Pi_m$. Alle zufälligen Partitionen auf $\mathbb{N}$ sind konsistent.
    Äquivalent dazu folgt aus der Definition der EPPF (\ref{Austauschbarkeit}), dass eine symmetrische Funktion  $p: \cup_{i=1}^{\infty} \mathbb{N}^k \to [0,1]$ existiert, mit 
    \begin{enumerate}
        \item $p(1) = 1$
        \item $\forall m < n$ und für alle Kompositionen $(n_1,...n_k)$ von $m$, gilt 
        \[ 
        p(n_1, ... n_k) = \sum_{i=1}^{k}p(n_1, ..., n_i + 1, ..., n_k) + p(n_1,...,n_k,1)
        \]
    \end{enumerate}
\end{Definition}
\begin{Definition}
    Ein Block $N_{n,i}$ einer Partition von $[n]$ besitzt eine \textit{assymptotische Frequenz}, falls der Limes 
    \[
    P_i := \lim_{n \to \infty}\frac{|N_{n,i}|}{n}
    \] 
    existiert
\end{Definition}
\textcolor{red}{add definition of ranked atoms,  De Finetti and page 43 of Pitman.}
\\
Wir beweisen nun eine analoge Aussage zu De Finetti für Partitionen von $\mathbb{N}$.
\textcolor{red}{Goal is to add De Finetti, Glivenko Cantelli, prove Kingman's representation, for next week to add stick breaking scheme PD dirichlet distribution and GEM distributions, then connect EPPF to CRP somehow}
\begin{lemma}
    \cite[Lemma 8.11]{NeiningerHS}
    \label{exchangeable expectations}
    \textcolor{red}{add notation explanation}
    Es sei $X = (X_{n})_{n \in \mathbb{N}}$ eine Folge austauschbarer Zufallsvariablen mit Werten in $E$ und $\phi: E \to \mathbb{R}$ messbar mit $\mathbb{E}[|\phi(X)|]< \infty$. Dann gilt für alle $n \geq l$ und $\pi \in S_n$
    \begin{enumerate}
        \item $ \mathbb{E}[\phi(X)|\varepsilon_n] =\mathbb{E}[\phi(X^\pi)|\varepsilon_n]  $  $\mathbb{P}$-f.s.
        \item $ \mathbb{E}[\phi(X)|\varepsilon_n] = \dfrac{1}{n!}\sum_{\pi \in S_n} \phi(X^\pi) = A_n(\phi)(X)$ $\mathbb{P}$-f.s.
    \end{enumerate}
\end{lemma}
\textcolor{red}{Form diesen Satz um} Wir benötigen lediglich den Fall $l = 1$, denn wir wollen auf die empirische Verteilungsfunktion schließen. In diesem Fall vereinfacht sich die zweite Gleichung von 2. auf $\frac{1}{n}\sum_{i=1}^{n}\phi(X_i)$. Mit dem darauffolgenden Satz in \cite{NeiningerHS}
\begin{theorem}
\label{exchangeable law of large numbers}
\cite[Satz 8.12]{NeiningerHS}
Sei $(X_n)_{n \in \mathbb{N}}$ austauschbar mit Werten in $E$ und $\phi: E^k \to \mathbb{R}$ eine messbare Funktion mit $\mathbb{E}[|\phi(X_1,X_2 ... X_k)|] < \infty$. Dann gilt
\[
\lim_{n \to \infty}A_n(\phi)(X) = \mathbb{E}[\phi(X)|\varepsilon] = \mathbb{E}[\phi(X)|\tau_\infty]\hspace{5pt} \mathbb{P} \hspace{5pt} \text{-f.s. und in } L_1
\] 
\end{theorem}
Falls wir $\phi := \chi_{(-\infty,x]}$ für ein $x \in \mathbb{R}$ definieren, dann bekommen wir 
\[ \lim_{n \to \infty} \frac{1}{n}\sum_{i=1}^{n} \chi_{(-\infty,x]}(X_i) =  \lim_{n \to \infty} \frac{1}{n}|\{i: X_i \leq x, i \leq n\}| = \mathbb{P}(X_1 \leq x |\varepsilon) \hspace{5pt}\mathbb{P}\hspace{5pt} \text{-f.s.}
\]
Wir definieren $F(x):= P(X_1 \leq x | \varepsilon)$ und verallgemeinern auf k-dimensionale Verteilungsfunktionen. Dazu sei $\phi:= \chi_{(-\infty,x_1]}\chi_{(-\infty,x_2]}...\chi_{(-\infty,x_k]}$ und wir bekommen analog zum 1-dimensionalen Fall aus Lemma \ref{exchangeable expectations} zusammen mit Satz \ref{exchangeable law of large numbers}
\begin{align*}
\lim_{n \to \infty} \dfrac{1}{n!}\sum_{\pi \in S_n} \phi(X^\pi) &= \lim_{n \to \infty}\dfrac{(n-k)!}{n!}\sum_{j_1=1}^{n} \sum_{j_2=1}^{n}...\sum_{j_k=1}^{n} \phi(X_{j_1},X_{j_2}...,X_{j_k}) \\
&= \lim_{n \to \infty} \frac{1}{n^k} \sum_{j_1=1}^{n} \phi(X_{j_1})\sum_{j_2=1}^{n} \phi(X_{j_2})...\sum_{j_k=1}^{n} \phi(X_{j_k})\\
&= \prod_{i=1}^{k}F(x_i).
\end{align*}
Daraus folgt also
\[
\mathbb{P}(X_1 \leq x_1, X_2 \leq x_2..., X_k \leq x_k| \varepsilon) = \prod_{i=1}^k F(x_i).
\]
Da nun $F(x)$ aber für alle $x$ $\varepsilon$ messbar ist, folgt mit der Turm Eigenschaft:
\[
\mathbb{P}(X_1 \leq x_1, X_2 \leq x_2..., X_k \leq x_k|F]  
= \mathbb{E}[\mathbb{P}(X_1 \leq x_1, X_2 \leq x_2..., X_k \leq x_k| \varepsilon)|F] = \prod_{i=1}^k F(x_i).
\]


Sei $\mu$ eine beliebige Verteilung mit Werten in $[0,1]$ und $(X_i)_{i \in \mathbb{N}}$ eine Folge von unabhängig identisch verteilten Zufallsvariablen mit Verteilung $\mu$.
\begin{theorem}
    Sei $\Pi_\infty$ eine austauschbare zufällige Partition von $\mathbb{N}$ und $(N^{\downarrow}_{n,i}, i \geq 1)$ die fallende Folge der Blockgrößen von $\Pi_n$, mit $N_{n,i}^\downarrow = 0$, falls $\Pi_n$ weniger als $i$ Blöcke hat. Dann konvergiert $\frac{N_{n,i}^\downarrow}{n} \rightarrow P^\downarrow_i$ P-f.s. für $n \to \infty$ $ \forall i \geq 1$. Weiter ist die bedingte Verteilung von $\Pi_\infty$ gegeben $(P_i^\downarrow, i \geq 1)$ gleich der Verteilung die durch zufälliges Sampling von einer zufälligen Distribution mit \textcolor{red}{ranked atoms} $(P^\downarrow_i, i \geq 1)$ entsteht.
\end{theorem}
\end{section}
\begin{section}{Chinese Restaurant Process}
% LTeX: language=de-DE
    Der \glqq Chinese Restaurant Process\grqq, dient der Konstruktion von zufälligen austauschbaren Partitionen, mithilfe welcher die Analyse von Bäumen erleichtert wird. In Kapitel \textcolor{red}{add something}, wird eine Bijektion zwischen den Bäumen und Partitionen aufgestellt und dies erlaubt uns, die in diesem Kapitel entwickelte Theorie anzuwenden. Wir beginnen mit der Konstruktion vom \glqq Chinese Restaurant Process\grqq.\\
\newline
\textbf{\fontsize{14}{18}\selectfont Konstruktion von CRP}
\\
Bevor die Konstruktion vom Prozess beschrieben wird, wollen wir vorab das Ziel der Konstruktion erläutern. Die Konstruktion erlaubt uns eine konsistente Folge von austauschbaren zufälligen Partitionen von $[n]$ zu erzeugen, also eine zufällige Partition $\Pi_\infty := (\Pi_n)_{n \in \mathbb{N}}$ von $\mathbb{N}$ zu generieren. Dies erlaubt uns u.a. Kingmans Sätze wie \textit{Kingman's Correspondance} und \textit{Kingman's representation} zu nutzen und somit Aussagen über die Grenzwerte der Größe der Blöcke von Partitionen zu formulieren. Wir beginnen mit einer einfachen Form des Prozesses: \\
\\
Wir betrachten eine Folge von zufälligen Permutationen $ \{\sigma_n, n \in \mathbb{N}\}$, sodass 
\begin{enumerate}
    \item $ \{\sigma_n\}$ Uniform verteilt auf $[n]$ ist.
    \item $\forall n \in \mathbb{N}$, falls $\sigma_n$ als Produkt von Zyklen dargestellt wird, $\sigma_{n-1}$ von $\sigma_n$ durch Deletion von Element $n$ zurückgewonnen wird.   
\end{enumerate}
Die Verteilung, die durch diese zwei Regeln entsteht, kann durch einen zufälligen Sitzplan in einem chinesischen Restaurant beschrieben werden. Man stelle sich das Restaurant zunächst vor als eine Sammlung von abzählbar unendlichen vielen Tischen, wo jeder Tisch die Sitzplatzkapazität von abzählbar unendlich vielen Kunden hat. Die Kunden, nummeriert mit den natürlichen Zahlen nach Reihenfolge Ihres Erscheinens, werden nach folgenden Regeln platziert.
\begin{enumerate}
    \item Der erste Kunde sitzt an Tisch 1
    \item Falls zum Zeitpunkt des Eintritts des $i$-ten Kundes $k$ Tische besetzt sind, so setzt sich dieser mit gleicher Wahrscheinlichkeit $(\frac{1}{n+1})$ links von Kunden $i: i \in \{1...n\}$ oder an einem neuen Tisch $k+1$.
\end{enumerate}
Wir definieren nun $\sigma_n: [n] \rightarrow [n]$ mithilfe dieser Regeln, sodass falls Kunde $i$ links von Kunde $j$ sitzt, $\sigma_n(i) = j$ und falls Kunde $i$ alleine sitzt $\sigma_n(i) = i$. Dass diese Folge von
Permutationen die ersten 2 Regeln befolgt, ist durch eine triviale Induktion ersichtlich. 
\\
Die Partitionen die durch die Zyklen von den Permutationen generiert werden, 
    %\item Falls zum Zeitpunkt des Eintritts des $i$-ten Kundes $k$ Tische besetzt sind, so setzt sich dieser an Tisch $j \in {1...k}$ mit Wahrscheinlichkeit $\frac{n_i-\alpha}{n+\theta}$ und an Tisch $k+1$ mit Wahrscheinlichkeit $\frac{\theta + k\alpha}{n + \theta}$
\end{section}
\chapter{Linear preferential trees} 
% LTeX: language=de-DE
%Das Kapitel beschreibt den Aufbau von \text{linear preferential trees} und wie diese zu Split Bäume 
In diesem Kapitel werden \textit{general preferential trees} eingeführt und die Einbettung von diesen in Devroyes \textit{Split} Bäume besprochen. Wir beginnen mit der Konstruktion von den jeweiligen Bäumen.
\\
\\
\textbf{\fontsize{14}{18}\selectfont Konstruktion von general preferential trees}\\
Der Aufbau von \textit{general preferential attachment trees} folgt im Wesentlichen dem Aufbau von \textit{random recursive Trees}. Wobei in \textit{Random recursive Trees} Knoten aufeinanderfolgend, mit uniformer Wahrscheinlichkeit als Kind von bereits existierenden Knoten angefügt werden, ist das Anknüpfen von Knoten von \textit{general preferential attachment trees} noch abhängig vom Ausgangsgrad von den existierenden Knoten. Präziser formuliert, wird der Vaterknoten $v$ von einem neuen Knoten, proportional zu $w_{d(v)}$ ausgesucht, wobei $d(v)$ der Ausgangsgrad vom Knoten $v$ ist, und $w_1, w_2...$ eine Folge von Gewichten ist. Der Fall $w_i = 1, \forall i \in \mathbb{N}$, führt somit genau auf den Fall von \textit{random recursive trees} zurück. In dieser Ausarbeitung werden die Gewichte von der Form 
\[
w_k = \chi k + \rho \hspace{5pt}, \chi \in \mathbb{R}\hspace{5pt} \rho > 0
\]
betrachtet, woher der Name \textit{linear preferential attachment trees} auch stammt. Da der Vaterknoten lediglich proportional zu der Folge von Gewichten ausgesucht wird, ändert die Skalierung der Folge nicht die Generierung der Baumstruktur, also beschränken wir die Betrachtung von den Gewichten auf 3 Spezialfälle:
\begin{enumerate}
    \item $\chi = 1$ ist der Fall vom \textit{general plane oriented recursive trees} \cite{panholzer2007level} 
    \item $\chi = 0$ ist der Fall vom \textit{random recursive trees}
    \item $\chi = -1$ ist der Fall von \textit{m-ary increasing trees}
\end{enumerate}
\begin{theorem}
\cite[Lemma 3.1]{janson2019random}
\label{Gewichte Satz}
   Mit den linearen Gewichten $w_k = \chi k + p $, hat ein Baum $T$ mit $m$ Knoten ein Gesamtgewicht $w(T) = (m-1)\chi + m\rho = m(\chi + \rho) - \chi$.
\end{theorem}
\begin{proof}
    Seien $d_1...d_m$ die Ausgangsgrade der jeweiligen Knoten im Baum. Dann ist die Summe der Ausgangsgrade, $\sum_{i=1}^{m}d_i = m-1$ und somit:
    \[
     w(T) = \sum_{i=1}^{m}w_i = \sum_{i=1}^{m} (\chi d_i + \rho) = m \rho + \chi \sum_{i = 1}^{m} d_i = m(\chi + \rho) - \chi  
    \]
\end{proof}
\begin{theorem} \cite[Theorem 3.2]{janson2019random}
    \label{linpreft Grenzwert}
    Sei $(T_n)_1^\infty =(T_n^{\chi,\rho})_1^\infty = $ eine Folge von \linpreft, mit den Kindern von der Wurzel in der Reihenfolge ihres Erscheinens beschriftet. Sei $N_j(n) := |T^j_n|$, die Größe des j-ten \PsubT von $T_n$. Dann gilt 
    \[ 
        \forall j \geq 1  \hspace{6pt} \frac{N_j(n)}{n}  \rightarrow P_j \hspace{6pt} \text{für} \hspace{6pt} n \rightarrow \infty \hspace{6pt} \text{P-f.s.}.
  \]
Wobei die Zufallsvariablen  $P_j$ Verteilung $GEM(\frac{\chi}{\chi+\rho},{\frac{\rho}{\chi+\rho}})$ haben. \textcolor{red}{add case 0} 
\end{theorem}
\begin{proof}
    Da wir im Allgemeinen $\chi + \rho > 0 $ annehmen dürfen, und da die Multiplikation von allen $w_k$ mit einer Konstante nicht den Baum ändert, können wir o.B.d.A   $\chi + \rho = 1 $ annehmen. Das Argument folgt nun durch Übersetzung von \PsubT zu Zyklen in Permutationen und eine Bijektion zwischen \linpreft und Permutationen. Wir beginnen mit der Bijektion. \\
    Dafür beschriften wir die Knoten sukzessiv, beginnend mit der Wurzel als 0, nach der Reihenfolge vom Hinzufügen zum Baum (der $i$-ter Knoten der hinzugefügt wird, wird mit $i-1$ beschriftet). Die Wurzel, die mit 0 beschriftet wird, wird identifiziert mit der leeren Permutation und der Zweite Knoten, welcher mit 1 beschriftet wird, wird mit dem Zyklus $(1)$ identifiziert. Wir identifizieren nun unsere Bäume mit Permutationen induktiv, nach folgenden zwei Regeln:
    \begin{enumerate}
        \item Falls der Knoten mit Beschriftung $i$ an die Wurzel angehängt wird, so fügen wir zur bestehenden Zyklus-Dekomposition, die Fixpunkt-Permutation $(i)$ hinzu.
        \item Falls der Knoten mit Beschriftung $i$ an ein anderen Knoten mit Beschriftung $j$, $j \neq 0$ angehängt wird, so fügen wir in den Zyklus $C$, welcher $j$ enthält, $i$ direkt nach $j$ hinzu. Bedeutet also, falls im ursprünglichen Zyklus $j \rightarrow k$ abgebildet wurde, im nächsten Schritt $j \rightarrow i \rightarrow k$ abgebildet wird.
    \end{enumerate}
Es ist einfach zu zeigen, dass diese Abbildung von Bäumen zu Permutation, welche durch diese Konstruktion entsteht, bijektiv ist und somit können wir die Theorie der \textit{austauschbaren Partitionen} anwenden, insbesondere also Satz \ref{Main theorem CRP}. Dafür betrachten wir die bedingten Wahrscheinlichkeiten, dass der $n+1$-Knoten an principal Subtree $i$ angefügt wird, gegeben er enthält bereits $n_i$ Knoten. Nach Lemma \ref{Gewichte Satz}, hat dieser Teilbaum Gesamtgewicht $n_i(\chi + \rho) - \chi = n_i - \chi$ und der gesamte Baum Gewicht $m-\chi$. Die bedingten Wahrscheinlichkeiten entsprechen also denen in der Konstruktion von einem $(\chi,\rho)$ \textit{Chinese Restaurant Process} und die Aussage folgt direkt aus Satz \ref{Main theorem CRP}.
\end{proof}
\begin{Bemerkung}
    Satz \ref{linpreft Grenzwert} wird eine zentrale Rolle bei der Bestimmung des \textit{Split Vektors} in der Einbettung von \textit{linear preferential attachment trees} in Devroyes Split Bäume spielen. Wir führen zuerst den Begriff und die Konstruktion der Split Bäume ein und kommen dann schließlich zu dieser Einbettung.  
\end{Bemerkung}
\textbf{\fontsize{14}{18}\selectfont Konstruktion von Split Bäumen}\\
Sei $b \geq 2$ fest und $\mathcal{P} = (P_i)_{1 \leq i \leq b}$ ein zufälliger Wahrscheinlichkeitsvektor, also $P_i \geq 0$ und $\sum_{i=1}^{b}P_i = 1$. Sei $\mathcal{T}_b$ der unendliche Wurzelbaum, in dem jeder Knoten $b$ Kinder hat, beschriftet mit $1,...,b$ und gebe jedem Knoten $v \in \mathcal{T}_b$ eine unabhängige Kopie von $\mathcal{P}$, die mit $\mathcal{P}^{(v)}$ bezeichnet wird. Jeder Knoten $v \in T_b$ kann maximal einen Knoten enthalten und falls dieser einen enthält, so nennen wir den Knoten \textit{voll}. Anfangs sind die Knoten leer und Bälle, beginnend mit der Wurzel, befüllen die Knoten nach folgenden zwei Regeln.
\begin{enumerate}
    \item Ein Ball, der an einem leeren Knoten ankommt, bleibt dort. Dieser Knoten ist dann \textit{voll}.
    \item Ein Ball, der an einem \textit{vollen} Knoten $v$ ankommt, bewegt sich weiter an die Kinder von $v$. Das Kind wird zufällig ausgewählt, wobei Kind $i$ mit Wahrscheinlichkeit $P_i^{(v)}$ ausgesucht wird.
\end{enumerate}





Mit dieser Vorarbeit, können wir nun die \textit{linear preferential attachment trees} in die Split Bäume einbetten.
\begin{theorem}
    Seien $(\chi,\rho)$ Parameter, welche die Voraussetzungen vom \textit{Chinese Restaurant Process} erfüllen und $\chi + \rho > 0$. Dann, falls die Bäume als ungeordnete Bäume betrachtet werden, hat der \textit{linear preferential attachment tree} $T_n^{\chi,\rho}$ für alle $n \in \mathbb{N}$ die selbe Verteilung wie der zufällige Split Baum $T^\mathcal{P}_n$ mit $b = \infty$ und 
    \[ 
    \mathcal{P} \sim GEM(\frac{\chi}{\chi + \rho}, \frac{\rho}{\chi + \rho})
    \] 
\begin{proof}
    Die Hauptidee hinter der Beweis liegt darin, die Konstruktion der \textit{linear preferential attachment trees} in einer rekursiven Form aufzufassen und dann die Verteilung der 
\end{proof}
\end{theorem}


\begin{section}{Verallgemeinerte CRP}
% LTeX: language=de-DE
Nachdem $n$ Gäste $k$ Tische besetzt haben, besetzt ein neuer Kunde einen Tisch nach folgenden Regeln:
\begin{enumerate}
    \item Der Gast besetzt ein Tisch $1 \leq i \leq k$ mit Wahrscheinlichkeit $\frac{n_i-\alpha}{n + \theta}$ 
    \item Der Gast besetzt einen neuen Tisch $k+1$ mit Wahrscheinlichkeit $\frac{\theta + k\alpha}{n + \theta}$.
\end{enumerate}
Damit diese tatsächlichen Wahrscheinlichkeiten entsprechen, müssen $\alpha$ und $\theta$ folgende Wahrscheinlichkeitsregeln erfüllen. \\
Das Haupttheorem ist nun folgendes.
\begin{theorem}
    Für jede der Parameter $(\alpha,\theta)$, welche die oben genannten Voraussetzungen erfüllen, generiert das CRP eine austauschbare zufälligen Partition von $\mathbb{N}$. Die dazu gehörende EPPF (siehe Def. 3) ist 
    \begin{equation}
   p_{\alpha,\theta}(n_1,...,n_k) = \frac{\displaystyle \prod_{i=0}^{k-2}(\theta + \alpha + i\alpha)\prod_{j=1}^{k}\prod_{r=1}^{n_i-1}(r-\alpha)}{\displaystyle\prod_{i=1}^{n-1}(\theta  + i)}.
    \end{equation}
    Die zugehörigen asymptotischen Frequenzen, in \textcolor{red}{size biased order of least elements} kann folgendermaßen repräsentiert werden.
    \[
    (\tilde{P_1},\tilde{P_2},\tilde{P_3}...) = (W_1,\bar{W_1}W_2,\bar{W_1}\bar{W_2}W_3...).
    \]
    Wobei $W_i$ unabhängig voneinander, nach $\beta(1 - \alpha, \theta + i\alpha)$ verteilt sind und $\bar{W_i}:= 1 - W_i$.
\end{theorem}  
\begin{proof}
    
\end{proof}

\end{section}
\chapter{Anwendungen}
% LTeX: language=de-DE
\begin{Definition}
    \label{definition height trees}
    Für einen gewurzelten Baum $T$, sei $h(v)$ die Tiefe eines Knotens $v$ und sei $v\wedge w$ der letzte gemeinsame Vorfahren von $v$ und $w$. Dann definieren wir die Funktion $Y$ als:
    \[
        Y := Y(T) = \sum_{v \neq w}h(v \wedge w),
    \]
    wobei wir die Summe über geordnete Paare $(v,w)$ definieren. Die Summe über geordnete Paare ist durch Symmetrie von $v \wedge w$ gleich dem doppelten der Summe über ungeordnete Paare. 
\end{Definition}
\begin{theorem}\textnormal{Hoeffdings Gesetz großer Zahlen für U-Statistiken}\\
    \label{hoeffding theorem}
    Seien $X_i$ i.i.d Zufallsvariablen, $f(X_1,...X_r)$ eine Funktion mit Erwartungswert $\theta$ und sei $\tilde{f}_n$ definiert durch 
    \[
        \frac{1}{(n)_{k}}\sum_{\substack{i_1,...,i_k \in \{1...n\}\\ \textnormal{paarweise verschieden}}}f(X_{i_1},...,X_{i_k}),
    \]
    wobei $(n)_k := \prod_{i=0}^{k-1}(n-i)$.
    Dann konvergiert 
    \[
        \tilde{f}_n  \rightarrow \theta \hspace{5pt} \textnormal{P-f.s.} \hspace{5pt} \textnormal{für } n \rightarrow \infty.
    \]
\begin{proof}
    Das Gesetz wird in \cite{hoeffding1961strong} bewiesen.
\end{proof}
\end{theorem}
\begin{theorem}
    \label{big Q theorem}
    Sei $(T_n^\mathcal{P})$ ein zufälliger Split Baum für einen \textit{Split Vektor} $\mathcal{P} = (P_i)_{i \in \mathbb{N}}$ und sei $Y_n := Y(T^\mathcal{P}_n)$ definiert wie in \ref{definition height trees}. Wir nehmen an, dass für ein $i \in \mathbb{N}$, $0< P_i < 1$ mit positiver Wahrscheinlichkeit. Dann existiert eine Zufallsvariable $Q$, sodass $Y_n/n^2 \rightarrow Q$ P-f.s. für $n \to \infty$. Weiter gilt 
    \begin{equation}
        \label{first equation of Q theorem}
    \mathbb{E}[Q]  = \frac{1}{1-\mathbb{E}[\sum_{i \geq 1}P_i ]}-1 < \infty. 
    \end{equation} 
    Zusätzlich erfüllt $Q$ die Verteilungs-Fixpunktgleichung (Distributional Fixed point equation)
    \begin{equation}
        Q \stackrel{\mathcal{L}}{=} \sum_{i=1}^{\infty}P_i^2(1+Q^{(i)}),
        \label{second equation of Q theorem}
    \end{equation} 
    wobei die $Q^{(i)}$ unabhängig voneinander und von $(P_i)_{i \in \mathbb{N}}$ sind, und $Q^{(i)} \stackrel{\mathcal{L}}{=}Q$. 
    Weiter gilt, mit $W$ so definiert wie nach Bemerkung \ref{Bemerkung PD Verteilungen}, dass 
\begin{equation}
    \mathbb{E}[Q] = \frac{\mathbb{E}[W]}{1 - \mathbb{E}[W]}.
        \label{third equation of Q theorem}
\end{equation}
\end{theorem}
\begin{proof}
  Wir beginnen mit einer Approximation von $Y_n$ für Split Bäume. Dafür modifizieren wir die Konstruktion von Split Bäumen und nehmen an, dass Anfangs bereits jeder Knoten \textit{voll} ist. Bedeutet, dass bereits der erste Knoten nicht an der Wurzel hängen bleibt, sondern sich anhand der jeweiligen \textit{Split Vektoren} der Knoten unendlich Tief entlang des Baums bewegt. Wir notieren den Pfad vom k-ten Ball, der sich so entlang des Baumes bewegt mit $\textbf{X}_k = (X_{k,i})_{i \in \mathbb{N}}$, wobei $X_{k,i} \in \mathbb{N}$ die Knotenbeschriftung vom Pfad in der  Tiefe $i$ bezeichnet. Gegeben \textit{Split Vektoren} $\mathcal{P}^{v}$ von allen Knoten $v$, sind die $\textbf{X}_i$ i.i.d. durch die Verteilung 
  \begin{equation}
    \label{independency sequence depth}
    \mathbb{P}(\bigcap_{j = 1}^{m}X_{k,j} = i_j) = \prod_{j=1}^{m}P_{i_j}^{(i_1...i_{j-1})}
  \end{equation}
  festgelegt, wobei wir mit $(i_1...i_{j-1})$ den Knoten in der $(j-1)$-ten Tiefe vom Pfad identifizieren. Sei nun für 2 Sequenzen $\textbf{X}^1,\textbf{X}^2 \in \mathbb{N}^{\infty}$ die Funktion $f$ folgendermaßen definiert:
  \[
    f(\textbf{X}^1,\textbf{X}^2) = min(\{i: \textbf{X}^1_i \neq \textbf{X}^2_i\}) - 1,
  \]
  also die Länge der längsten gemeinsamen Anfangssequenz. Dann ist 
  \[
    f(\textbf{X}^k,\textbf{X}^l) = h(v_k,v_l),
  \]
  falls beide gegenseitig kein Vorfahren vom anderen sind und $\textbf{X}^k$ bzw. $\textbf{X}^l$ die zugehörende Sequenz vom Knoten $v_k$ bzw. $v_l$ ist. In dem Falle das $v_l$ ein Vorfahren von $v_k$ ist, gilt 
  \begin{equation}
    \label{inequality for height}
    f(\textbf{X}^k,\textbf{X}^l) \geq h(v_k,v_l).
  \end{equation} \\
  Mithilfe dieser Definition definieren wir unsere Approximation durch 
  \[
    \tilde{Y}_n = 2\sum_{l < k \leq n}f(\textbf{X}^k,\textbf{X}^l).
  \]
Bedingt auf alle \textit{Split Vektoren} $\mathcal{P}^{v}$, bekommen wir mit $\chi$ definiert als Indikator Funktion. \textcolor{red}{Warum verschwindet die Bedingung?}
\begin{align}
\mathbb{E}[f(\textbf{X}^1,\textbf{X}^2)|\{\mathcal{P}^{v}, v \textnormal{ Knoten}\}] &=\sum_{m=1}^{\infty}\mathbb{P}({f(\textbf{X}^1,\textbf{X}^2) \geq m}|\{\mathcal{P}^{v}, v \textnormal{ Knoten}\}) \nonumber\\
&=\sum_{m=1}^{\infty}\mathbb{P}(\bigcap_{j=1}^{m}\textbf{X}_{1,j}=\textbf{X}_{2,j})   \nonumber\\
&=\sum_{m=1}^{\infty}\mathbb{P}(\bigcup_{i_1,...,i_m \in \mathbb{N}} \{\bigcap_{j=1}^{m}\textbf{X}_{1,j}=\textbf{X}_{2,j} = i_j) \}   \nonumber\\
&=\sum_{m=1}^{\infty}\sum_{i_1,...,i_m \in \mathbb{N}}\mathbb{P}(\{\bigcap_{j=1}^{m}\textbf{X}_{1,j}=\textbf{X}_{2,j} = i_j \}) \nonumber\\
&=\sum_{m=1}^{\infty}\sum_{i_1,...,i_m \in \mathbb{N}}\mathbb{P}(\bigcap_{j=1}^{m}\textbf{X}_{1,j} = i_j )\mathbb{P}(\bigcap_{j=1}^{m}\textbf{X}_{2,j} = i_j ) \nonumber\\
&=\sum_{m=1}^{\infty}\sum_{i_1,...,i_m \in \mathbb{N}}(\prod_{j=1}^{m}P_{i_j}^{(i_1...i_{j-1})})^2 =: Q \label{Gleichungskette Tiefe}.
\end{align}
Mit der Turmeigenschaft zusammen mit der Voraussetzung, dass die $P^{(v)}$ i.i.d sind, folgt 
\begin{align}
\mathbb{E}[f(\textbf{X}^1,\textbf{X}^2) ] = \mathbb{E}[Q] &=\sum_{m=1}^{\infty}\sum_{i_1,...,i_m \in \mathbb{N}}\prod_{j=1}^{m}\mathbb{E}[P_{i_j}^{2}] \nonumber\\
 &= \sum_{m=1}^{\infty}(\sum_{i =1}^{\infty}\mathbb{E}[P_{i}^{2}])^m \nonumber \\
 &= \frac{1}{1- \sum_{i=1}^{\infty}\mathbb{E}[P_i^2]}-1  = \frac{\sum_{i=1}^{\infty}\mathbb{E}[P_i^2]}{1- \sum_{i=1}^{\infty}\mathbb{E}[P_i^2]}  \nonumber.
\end{align}
Wobei wir in die letzte Gleichung die Summenformel für eine geometrische Reihe ausgenutzt haben, da $\sum_{i=1}^{\infty} P_i^2 \leq \sum_{i=1}^{\infty} P_i = 1$ mit positiver Wahrscheinlichkeit, $\sum_{i=1}^{\infty} \mathbb{E}[P_i^2] < 1$ impliziert. Die zweite Behauptung des Satzes (\ref{first equation of Q theorem}) ist somit gezeigt. Für die letzte Behauptung \ref{third equation of Q theorem}, nutzen wir 
\[
    P(W=P_i|(P_i)_{i \in \mathbb{N}}) = P_i.  
\]  
und folgern
\begin{align}
\mathbb{E}[W] &= \mathbb{E}[\mathbb{E}[W|(P_i)_{i \in \mathbb{N}}]] \\
&= \mathbb{E}[\sum_{i=1}^{\infty}P_i\mathbb{P}(W = P_i|(P_i)_{i \in \mathbb{N}})] \\
&= \mathbb{E}[\sum_{i=1}^{\infty}P_i^2].
\end{align}
Nach Hoeffdings Gesetz der großen Zahlen (Satz \ref{hoeffding theorem}), gilt
\[
    \frac{\tilde{Y}_n}{n(n-1)} \rightarrow Q \hspace{5pt} \textnormal{P -f.s.} \hspace{5pt} \textnormal{für } n \to \infty.
\]
Es fehlt zu zeigen, dass $(\tilde{Y}_n-Y_n)/n^2 \rightarrow 0 $ P-f.s. für $n \to \infty$. \\
Sei 
\begin{equation}
H_n := max\{h(v) : v \in T_n\}
\end{equation}
die Höhe von $T_n := T_n^\mathcal{P}$ und
\begin{equation}
H^*_n := max\{f(\textbf{X}_k,\textbf{X}_l) : l < k \leq n\}.
\end{equation}
Wir definieren $v \prec w$, falls $v$ ein Vorfahren von $w$ ist. Dann gilt durch \ref{inequality for height}
\begin{align}
    0 \leq \tilde{Y}_n - Y_n &= 2\sum_{k=1}^{n}\sum_{v_l \prec v_k}(f(\textbf{X}_k,\textbf{X}_l)-h(v_k \wedge v_l)) \nonumber\\
    &\leq 2\sum_{k=1}^{n}\sum_{v_l \prec v_k}f(\textbf{X}_k,\textbf{X}_l) \nonumber \\
    &\leq 2\sum_{k=1}^{n}\sum_{v_l \prec v_k}H_n^* \nonumber \\
    &\leq 2\sum_{k=1}^{n}H_nH_n^*  = 2nH_nH_n^* \label{Abschätzung Tiefe}
\end{align}
wobei die letzte Ungleichung Wahr ist, da ein Knoten höchstens $H_n$ Vorfahren hat. \\
Nun hat der tiefst-liegende Knoten $k$ die Tiefe $H_n$ und falls der Vaterknoten mit $l$ bezeichnet wird, so gilt 
\begin{equation}
    H^*_n \geq f(\textbf{X}^k,\textbf{X}^l) \geq H_n - 1. \label{trivial inequality}
\end{equation}
Sei $m := m_n := \lceil c \log(n) \rceil$. Dann gilt, wie in der Gleichungskette \ref{Gleichungskette Tiefe},
\begin{align*}
\mathbb{P}(f(\textbf{X}^1,\textbf{X}^2)\geq m| \{\mathcal{P}^v, v \textnormal{ Knoten}\}) &=\sum_{i_1,...,i_m \in \mathbb{N}} \mathbb{P}( \{\bigcap_{j=1}^{m}\textbf{X}_{1,j}=\textbf{X}_{2,j} = i_j \}) \\ 
&=\sum_{i_1,...,i_m \in \mathbb{N}}(\prod_{j=1}^{m}P_{i_j}^{(i_1,...,i_{j-1})}) ^2.
\end{align*}
Und somit
\[
\mathbb{P}(f(\textbf{X}^1,\textbf{X}^2)\geq m) = \sum_{i_1,...,i_m \in \mathbb{N}}\prod_{j=1}^{m}\mathbb{E}[P_{i_j}^2]  =: a^m,
\]
für $a := \sum_{i=1}^{\infty}\mathbb{E}[P_i^2] < 1$. Wir können jetzt $H^*_n$ abschätzen. Dafür wählen wir $c \geq 4 / |\log{a}|$ und rechnen
\begin{align*}
\mathbb{P}(H^*_n \geq m) &\leq \sum_{l < k \leq n} \mathbb{P}(f(\textbf{X}_k,\textbf{X}_l) \geq m) = {n\choose 2}\mathbb{P}(f(\textbf{X}_k,\textbf{X}_l) \geq m)\\
&\leq n^2\mathbb{P}(f(\textbf{X}_k,\textbf{X}_l) \geq m) \leq n^2a^m \leq n^2a^{c \log{n}} \leq n^{-2}.
\end{align*}
Da $\sum_{n \geq 1}\frac{1}{n^2} < \infty$, folgt mit Borel-Cantelli $H_n \leq c \log{n}$ P-f.s. für $n \to \infty$. Nun gilt mit \ref{trivial inequality} und \ref{Abschätzung Tiefe}, dass $(\tilde{Y}_n - Y_n) \in O(n\log^2(n))$ und somit ist die Behauptung $(\tilde{Y}_n-Y_n)/n^2 \rightarrow 0 $ P-f.s. gezeigt. 
\end{proof}
\begin{Korollar}
    Für ein \textit{linear preferential attachment tree} mit Parametern $(\chi,\rho)$ mit $\chi+ \rho > 0$ existiert eine Zufallsvariable $Q$, sodass $Y(T_n^{\chi,\rho})/n^2 \to Q$ P-f.s. Weiter hat $Q$ den Erwartungswert 
    \[
        \mathbb{E}[Q] = \rho.
    \]
\end{Korollar}
\begin{proof}
    Das $Q$ existiert folgt direkt aus Satz \ref{big Q theorem}. Nach Satz \ref{Verteilung W}, ist $W \sim \beta(\rho,1)$ und hat somit Erwartungswert $\frac{\rho}{\rho + 1}$. Satz \ref{big Q theorem} liefert nun
    \begin{equation*}
        \mathbb{E}[Q] = \frac{\mathbb{E}[W]}{1-\mathbb{E}[W]} = \frac{\frac{\rho}{\rho + 1}}{\frac{1}{\rho+1}} = \rho
    \end{equation*}
\end{proof}

\chapter{m-ary increasing trees}
% LTeX: language=de-DE
In diesem Kapitel werden die Teilmenge der \textit{linear preferential attachment trees}, mit Parametern der Form $\chi < 0$ und $\rho > 0$ näher betrachtet. Wie schon im ersten Kapitel bemerkt, ändert die Normierung von allen Gewichten $w_k$ mit einer Konstante $c \in \mathbb{R}$ nicht die Verteilung der Bäume. Infolgedessen, betrachten wir $\chi = -1$ und prüfen, bei welchen $\rho \in \mathbb{R}^{+}$ es zu negative Gewichten kommt.\\
Für $\chi = -1$ sind die Gewichte der \textit{linear preferential trees} durch 
\[
    w_k = \chi k + \rho =  \rho - k 
\]
festgelegt. Für $k > \rho$ folgt daraus, dass $w_k < 0$. Daher ist die einzige Möglichkeit negative Gewichte zu vermeiden, $w_m = 0$, für ein $m \in \mathbb{N}$ sicherzustellen. In diesem Fall kann einem Knoten mit $m$ Kindern kein weiterer Knoten hinzugefügt werden, und die weiteren Gewichte $w_n, n > m$ sind von keiner Relevanz. Auf $\rho$ bezogen, heißt das, dass $\rho = m\in \mathbb{N}$, falls $\chi = -1$, bzw. $\rho= m\chi, m \in \mathbb{N}$, falls wir $\chi$ nicht zu $-1$ normieren.\\
Bis jetzt wurden Bäume nur in einer ungeordneten Weise betrachtet, d.h. es zählte nur welche Geschwister ein Knoten hat, und nicht die Reihenfolge der Geschwister. Da in dieser Teilmenge der \textit{linear preferential attachment trees} die Anzahl der Kinder pro Knoten durch $m$ beschränkt ist, können wir die Bäume auch in einer geordneten Weise interpretieren. Das Beispiel $T_n^{-1,2}$ illustriert wie wir diesen Gedanken nachverfolgen können.
\begin{Beispiel}
    Die Gewichte von einem Knoten $v$ in einem $T_n^{-1,2}$ haben die Form: 
    \[
        w_k = 2 - k, k \in \{0,1,2\},
    \]
    wobei $k$ der Ausgangsgrad von $v$ ist. Wir können die Gewichte umschreiben, indem wir die Gewichte proportional zu den externen Kindern von $v$ auffassen. Es gibt $e = m - k = 2-k$ externe Kinder $e$ von einem Knoten $v$ und somit haben die Gewichte die Form 
    \[
        w_k = 2 -k = 2 - (2-e) = e, e \in \{0,1,2\}. 
    \]
    Das bedeutet jeder Knoten hat Gewicht direkt proportional zu der Anzahl der externen Knoten. Diese Eigenschaft erlaubt uns den Aufbau von $T_n^{-1,2}$ geordnet zu interpretieren. Dazu beginne man mit einem vollständigen unendlichen 2-ären Baum, mit Kindern von einem Knoten mit $\{1,2\}$ beschriftet und dann zufällig Uniform permutiert. Wir konstruieren dann den $(T_n^{-1,2})_{n \in \mathbb{N}}$ innerhalb von diesem $\mathcal{T}_2$ Baum: in jedem Schritt wird ein externer Knoten von $T_n^{-1,2}$ mit Beschriftung aus $\mathcal{T}_2$ uniform ausgewählt und hinzugefügt (Siehe Abbildung \ref{picture binary trees})
\end{Beispiel}

\usetikzlibrary{shapes.geometric}
\begin{figure}[h!]
\begin{center}
\begin{tikzpicture}[level/.style={sibling distance=50mm/#1}]
	

	\node[circle,draw] {W}
		child {node[circle,draw] {1}
			child {node[circle,draw,dashed]  {1}}
            child {node[circle,draw]  {2}
                child{node[circle,draw,dashed]{1}}
                child{node[circle,draw,dashed]{2}}}}
		child {node[circle,draw] {2}
			child {node[circle,draw]  {2}
                child{node[circle,draw,dashed]{2}}
                child{node[circle,draw,dashed]{1}}}
            child {node[circle,draw]  {1}
                child{node[circle,draw,dashed]{2}}
                child{node[circle,draw,dashed]{1}}}
		};
\end{tikzpicture}
\caption{\label{picture binary trees}Beispiel für eine Konstruktion von einem $T_n^{-1,2}$ Baum. Die externen Knoten sind mit gestrichelter Linien abgebildet und $W$ ist die Wurzel}.
\end{center}
\end{figure}
Die geordnete Interpretation von $T_n^{-1,2}$ entspricht daher einem zufälligen Binärsuchbaum. Diese sind aber einer der elementaren Beispiele von Split Bäumen behandelt in \cite{devroye1998universal}, welche durch den Split Vektor $\mathcal{P} = (U,1-U)$ und $U \sim Uniform[0,1]$ charakterisiert ist. Dies steht nicht in Widerspruch zu Satz \ref{main theorem paper}, denn obwohl die Bäume $T_n^{-1,2}$ und $T_n^{(U,1-U)}$ in der ungeordneten Interpretation dieselbe Verteilung haben, ist die Verteilung von $T_n^{\mathcal{P}}$ nicht eindeutig durch $\mathcal{P}$ festgelegt. Der Split Vektor $\mathcal{P}$ ist nach Bemerkung \ref{Bemerkung PD Verteilungen} nur bis auf (zufällige) Permutationen eindeutig, bedeutet es muss eine (zufällige) Permutation von $\mathcal{P}^1 := (P_1,1-P_1) \sim GEM(-1,2)$ geben, sodass $\mathcal{P}^1 \stackrel{\mathcal{L}}{=} \mathcal{P}^2 := (U,1-U)$. Nach Satz \ref{Main theorem CRP} ist $P_1 \sim \beta(1,2)$ und hat somit nach Definition \ref{definition beta distribution} Wahrscheinlichkeitsdichte $2x$ und $1-P_1$ hat Dichte $2-2x$. Falls wir annehmen, dass die Permutation unabhängig von $(P_1,1-P_1)$ ist, so kann man der Permutation Wahrscheinlichkeiten zuordnen, sodass $(P_1,1-P_1)$ mit Wahrscheinlichkeit $p_1$ und $(1-P_1,P_1)$ mit Wahrscheinlichkeit $p_2$ gewählt wird. Wir versuchen dann $p_1,p_2$ zu bestimmen, sodass wir die Verteilung $(U,1-U)$ zurückgewinnen. Wir definieren unsere zufällige Permutation mit $(\tilde{P}_1,\tilde{P}_2)$ und berechnen für $a \in [0,1]$
\begin{align*}
 \mathbb{P}(\tilde{P}_1 \leq a) &=  \mathbb{P}(P_1 \leq a| \tilde{P}_1 = P_1)\mathbb{P}(\tilde{P}_1 = P_1) + \mathbb{P}(P_1 \leq a| \tilde{P}_1 = P_2)\mathbb{P}(\tilde{P}_1 = P_2) \\
&= \mathbb{P}(P_1 \leq a| \tilde{P}_1 = P_1)p_1 + \mathbb{P}(P_1 \leq a| \tilde{P}_1 = P_2)p_2 \\
&= \int_{0}^{a}p_12x \textnormal{dx} + \int_{0}^{a}p_2(2-2x) \textnormal{dx}\\
&= \int_{0}^{a}p_12x + p_2(2-2x) \textnormal{dx}.
\end{align*}
Somit hat $\tilde{P}$ Wahrscheinlichkeitsdichte $p_12x + p_2(2-2x)$.
$U$ hat nach Voraussetzung Wahrscheinlichkeitsdichte $1$ und mit einem Koeffizientenvergleich erkennt man sofort, dass $p_1 = p_2 = \frac{1}{2}$ sein muss, damit die Dichten Übereinstimmen. In anderen Worten stimmt die Verteilung von $(P_1,1-P_1) \sim GEM(-1,2)$ nach einer uniformen Permutation mit $(U,1-U)$ überein. Da $(P_1,P_2) \sim PD(-1,2)$ eine  absteigende Anordnung von $(P_1,P_2) \sim GEM(-1,2)$ (nach Definition \ref{GEM und PD Verteilungen}), ist $PD(-1,2)$ eine absteigende Anordnung und somit eine Permutation von $(U,1-U)$. Falls wir $(P_1,1-P_1) \sim PD(-1,2)$ definieren, können wir die Verteilung von $P_1$ mit $a \in [0,1]$ bestimmen 
\begin{align*}
    \mathbb{P}(max(U,1-U) \leq a) &= \mathbb{P}(U \leq a \cap 1-U \geq a)\\
    = \mathbb{P}(1-a \leq U \leq a) &= \mathbb{P}(U \leq 2a - 1)\\
    &= \begin{cases}
        0 & a < 0.5 \\
      2a-1  & 0.5\leq a\leq 1\\
      1 & 1  > a 
    \end{cases}
\end{align*}
und somit ist $PD(-1,2) \stackrel{\mathcal{L}}{=} U(\frac{1}{2},1)$. \\

Die Betrachtungsweise von $T_n^{-1,m}$, als geordnete Bäume lässt sich auch auf $m > 2$ verallgemeinern. Diesmal betrachten vollständige $\mathcal{T}_m$ Bäume, mit Kindern von einem Knoten mit $\{1,...,m\}$ beschriftet und dann zufällig Uniform permutiert, wieder wird $(T_n^{-1,m})_{n \in \mathbb{N}}$ in $\mathcal{T}_m$ eingebettet und in jedem Schritt ein externer Knoten von $(T_n^{-1,m})_{n \in \mathbb{N}}$ mit Beschriftung in $\mathcal{T}_m$ uniform ausgewählt und hinzugefügt. Diese geordnete Konstruktion von einem Baum wird mit \textit{m-ary increasing tree} bezeichnet. Wir bestimmen jetzt den Split Vektor welcher den \textit{m-ary increasing tree} auch für $m >2$ charakterisiert.
\begin{theorem}
    \label{Satz mary increasing trees}
    Sei $m \geq 2$. Die Folge von \textit{linear preferential attachment trees} $(T_n^{-1,m})_{n \in \mathbb{N}}$, betrachtet als m-ary increasing Trees haben dieselbe Verteilung wie die Folge der Split Bäume $(T^\mathcal{P}_n)_{n \in \mathcal{N}}$, mit 
    \[
    \mathcal{P} = (P_i)_{i \in \mathbb{N}} \sim Dir(\frac{1}{m-1},...,\frac{1}{m-1}),
    \]
    wobei mit $Dir$ die Dirichlet-Verteilung gemeint ist.
\end{theorem}

\begin{Definition}\textnormal{Dirichlet-Verteilung}\\
    Ein Wahrscheinlichkeitsvektor $(X_1,...,X_m)$ ($\sum_{i=1}^{m} X_i \geq 0$ und $X_i \geq 0$) der Länge $m$ ist $Dir(\alpha_1,...,\alpha_m)$ verteilt, falls die Wahrscheinlichkeitsdichte durch
    \[
    f(x_1,...,x_m) = \frac{\Gamma(\sum_{i=1}^{m}x_i)}{\prod_{i=1}^{m}\Gamma(x_i)} \prod_{i=1}^{m}x_i^{\alpha_i-1}
    \] 
    für $x_i \geq 0$ und $\sum_{i=1}^{m}x_i = 1$ und für sonstige $(x_1,...,x_m)$ durch $0$ gegeben ist.
\end{Definition}
\begin{proof}[Beweis von Satz \ref{Satz mary increasing trees}]
    Wir beginnen mit der Existenz vom Split Vektor $\mathcal{P}$.  Aus Satz \ref{main theorem paper} folgt, dass $T_n^{-1,m} \stackrel{\mathcal{L}}{=}T_n^{\mathcal{P}}$, mit $\mathcal{P} \sim PD(-\frac{1}{m-1},\frac{m}{m-1})$, falls wir beide Bäume als ungeordnet interpretieren. Zur Überführung in dem geordneten Fall betrachte man eine zufällige uniforme Permutation von $\mathcal{P}$, bezeichnet mit $\tilde{\mathcal{P}}$. In diesem Fall ist durch Symmetrie sowohl im ungeordneten als auch im geordneten Fall $T_n^{-1,m} \stackrel{\mathcal{L}}{=}T_n^{\tilde{\mathcal{P}}}$, denn beide Bäume sind invariant unter zufälligen Umbenennungen von allen Kindern der Knoten. \\
    Es fehlt zu zeigen, dass diese Permutation von einer $PD$ Verteilung, tatsächlich einer $Dir$ Verteilung entspricht. Wir definieren für ein \textit{principal subtree} $T^j_n$, $N_j(n) := |T^j_n|$ als Anzahl der Knoten und $N_j^e(n)$ als Anzahl der externen Knoten von $T_j^n$. Dann ist $N_j^e(n) = (m-1)N_j(n) + 1$ durch eine einfache Induktion, denn für jeden zusätzlichen Knoten der zu $T_j^n$ hinzukommt, kommen von diesem Knoten aus zusätzliche $m$ externe Knoten hinzu und ein externer Knoten wird durch den hinzugekommenen Knoten ersetzt. \\
    Wir interpretieren jetzt $T_n^{-1,m}$ wieder als ungeordneten Baum und identifizieren die externen Knoten im \textit{principal Subtree} $T^j_n$ als Bälle mit Farbe $j$. Die Anzahl der externen Knoten mit Farbe $j$ folgen jetzt einem Polya Urnen Prozess. Am Anfang gibt es in der Urne einen Ball von jeder Farbe (Der Baum besteht nur aus der Wurzel) und in jedem Schritt wird ein zufälliger Ball gezogen und anschließend mit $m-1$ zusätzlichen Bälle derselben Farbe in die Urne zurückgelegt.  Dieser Art von verallgemeinerten Polya Urnen Prozess wurde in \cite{athreya1969characteristic} untersucht und bezogen auf unsern Fall, konvergiert
    \[
        \left(\frac{N^{e}_j(n)}{(m-1)n+1}\right)_{j \in \{1...m\}} = \left(\frac{(m-1)N_j(n) + 1}{(m-1)n+1}\right)_{j \in \{1...m\}} \rightarrow (X_j)_{j \in \{ 1...m\}} \hspace*{5pt} P-f.s \hspace*{5pt} \textnormal{für } n \to \infty,
    \]
    mit $(X_j)_{j \in \{1...m\}} \sim Dir(\frac{1}{m-1},...,\frac{1}{m-1})$. \\
    Insbesondere folgt $N_j(n)/n \to (X_j)$ P-f.s. Dieser Grenzwert ist aber bereits der Grenzwert der unseren Wahrscheinlichkeitsvektor liefert. Denn ähnlich zu Satz \ref{Verteilung W}, liefert uns das Gesetz der großen Zahlen, gegeben den Split Vektor an der Wurzel,
    \[
        \frac{N_j(n)}{n} \rightarrow P_j \hspace{5pt} \textnormal{P-f.s.} \hspace*{5pt} \textnormal{für } n \to \infty \hspace*{5pt} \forall j \in \{1...m\}.
    \]
    In anderen Worten ist $(P_j)_{j \in \{1...m\}} \sim Dir(\frac{1}{m-1},...,\frac{1}{m-1})$ und die Behauptung ist gezeigt.

\end{proof}

\chapter*{Anhang}
% LTeX: language=de-DE
\begin{lemma}
\label{combinatorial lemma}
Die Anzahl der Partitionen von $[n]$ mit Klassengröße in Reihenfolge ihres Erscheinens ist durch \ref{combinatorial equation} gegeben.
\end{lemma}
\begin{proof}
    Das $\#(\textbf{n})$ tatsächlich die Form \ref{combinatorial equation} hat, lässt sich durch Betrachtung von
    \begin{align*}
   \Omega_{\textbf{n}} = \{ \textbf{x}:= (x_{n_1}^{1},x_{n_1}^{2},...,x_{n_1}^{n_1},x_{n_2}^{1},x_{n_2}^{2},...,x_{n_k}^{n_k})|& \textbf{n}:= (n_1,...,n_k), \sum_{i=1}^{k}n_i=  n \\
   &,\textbf{x} \in \mathcal{P}(n)\\
   &, x_{n_i}^j < x_{n_i}^{j+1}\hspace{5pt} \forall  i \in \{1...k\} \forall j \in \{1...n_i-1\} \\
   &, x^1_{n_i} < x^1_{n_i+1} \hspace{5pt} \forall i \in \{1...k\}\}
    \end{align*} 
    nachweisen. Es gilt
    \[
   \#(\textbf{n}) =  |\Omega_\textbf{n}|
    \]
    und wir berechnen nun $|\Omega_\textbf{n}|$. Es gibt $n!$ Möglichkeiten $[n]$ zu permutieren, nach den Bedingungen von $\Omega_n$ muss aber $x_{n_i}^1$ für alle $i$ kleiner sein als jede darauffolgende Zahl in der Liste. Formal beschrieben bedeutet das, falls $\textbf{x}_i, 1 \leq i \leq n$ die i-te Komponente von $\textbf{x}$ ist, so ist $\textbf{x}_{n_i} < \textbf{x}_{j}$ $\forall i \in \{n_i+1,...,n\}$. Sei  
    \begin{align*}
   \Omega^1_{\textbf{n}} = \{ \textbf{x}:= (x_{n_1}^{1},x_{n_1}^{2},...,x_{n_1}^{n_1},x_{n_2}^{1},x_{n_2}^{2},...,x_{n_k}^{n_k})|& \textbf{n}:= (n_1,...,n_k), \sum_{i=1}^{k}n_i=  n \\
   &,\textbf{x} \in \mathcal{P}(n)\\
   &, \textbf{x}_{n_i} < \textbf{x}_j \forall i \in \{1...k\},\forall j \in \{n_i+1,...,n\}\}
    \end{align*} 
    so ist
    \[
    |\Omega^1_{\textbf{n}}| = \frac{n!}{n(n-n_1)(n-n_2-n_1)...(n-\sum_{i=1}^{k-1}n_i)}.
    \]
    Um von $|\Omega_\textbf{n}^1|$ auf $|\Omega_\textbf{n}|$ zu kommen fehlt uns in jedem der einzelnen Komponenten $n_i$ die Permutationen der $n_i -1$ nachfolgenden Zahlen nach jedem der $\textbf{x}_{n_i}$ zu zählen. Da es aber $\prod_{i=1}^{k}(n_i-1)!$ davon gibt, gilt 
    \[
     |\Omega_\textbf{n}| = \frac{|\Omega_\textbf{n}^1|}{\prod_{i=1}^{k}(n_i-1)!}
    \]
    und die Behauptung ist gezeigt.
\end{proof}

%\section{Einleitung}
\bibliographystyle{plain}
\bibliography{Sources/Stochastik_Literatur}
\end{document}