\begin{section}{Definitionen}
% LTeX: language=de-DE
\begin{Definition}
   Sei $F$ eine endliche Menge. Dann ist eine Partition von $F$ eine ungeordnete Sammlung von nichtleeren, disjunkten Untermengen $\{A_i| 1 \leq i \leq k, A_i \subseteq F\}$, mit $\cup_{i=1}^{k}A_k = F$.\\
   Eine Partition von einer natürlichen Zahl $n$, ist eine ungeordnete Menge von positiven natürlichen Zahlen mit Summe $n$.
\end{Definition}
\begin{Definition}
    \textnormal{(Zufällige Partitionen)}
\begin{enumerate}
    \item Eine zufällige Partition von $n\in \mathbb{N}$ ist eine Zufallsvariable $\pi_n$ mit Werten in der endlichen Menge von allen Partitionen von $n$.
    \item Eine zufällige Partition von $[n] := \{1 ...n\}$, ist eine Zufallsvariable $\Pi_n$ mit Werten in der endlichen Menge von Partitionen von $[n]$.
    \item Eine zufällige Partition auf $\mathbb{N}$ ist eine Folge $\Pi = (\Pi_n)_{n\in \mathbb{N}}$ von Zufallsvariablen auf einem gemeinsamen Wahrscheinlichkeitsraum mit Werten in der endlichen Menge von Partitionen von $[n]$, sodass für $i<j, i,j \in \mathbb{N}$ die Restriktion von $\Pi_j$ auf $[i]$ $\Pi_i$ ist.
\end{enumerate}

\end{Definition}
\begin{Definition}
    \textnormal{(Austauschbarkeit)}
    \label{Austauschbarkeit}
    \begin{enumerate}
        \item Eine zufällige Partition $\Pi_n$ von $[n]$ heißt partiell Austauschbar, falls für jede Partition $\{ A_1 ... A_k\}$ von $[n]$ in Reihenfolge ihres Erscheinens\footnote{\textit{In Reihenfolge ihres Erscheinens} bedeutet $1 \in A_1$, und für alle $2 \leq i \leq k$, das erste Element von $[n]\setminus \cup_{j=1}^{i-1}A_j$ zu $A_i$ gehört}.
        \[ 
            \mathbb{P}(\Pi_n = \{ A_1 ... A_k\}) = p(|A_1|, ... |A_k|)
        \]  
        für eine Funktion p von Kompositionen $\textbf{n}:= (n_1, ..., n_k)$ von $n$. Diese Funktion wird auch PEPF (Partially Exchangeable Probability Function) bezeichnet.
        \item Falls die Funktion $p$ zusätzlich noch symmetrisch ist, so ist die zufällige Partition $\Pi_n$ Austauschbar und man bezeichnet $p$ als \textit{EPPF} (Exchangeable Partition Probability Function)
    \end{enumerate}
\end{Definition}

\begin{Definition}{Konsistenz}\\
    Wir nennen eine Folge von austauschbaren zufälligen Partitionen $(\Pi_n)$ \textit{konsistent}, falls die Restriktion $\Pi_{m,n}$ von $\Pi_m$ zu $[n]$ für alle $m > n$ die gleiche Verteilung hat wie $\Pi_m$. Alle zufälligen Partitionen auf $\mathbb{N}$ sind konsistent.
\end{Definition}
\begin{lemma}
    Sei $\Pi_n$ eine partiell austauschbare zufällige Partition von $[n]$ mit \textit{PEPF} $p(\textbf{n})$ und $\Pi_m$ die Restriktion von $\Pi_n$ auf $[m], m \in \{1...n\}$, dann gilt mit $\textbf{m} := (m_1,...,m_k)\in \bigcup_{i = 1 }^{n} \mathbb{N}^i$, $\sum_{i=1}^{k}m_i = m$ und $\textbf{m}^{i+} = \textbf{m} + e_i, i \in \{1...k+1\}$ für einen Einheitsvektor $e_i$:
\begin{center} 
    $\Pi_m$ ist partiell Austauschbar mit PEPF $p(\textbf{m}):=  \sum_{j=1}^{k+1} p(\textbf{m}^{j+})$.
\end{center}
\end{lemma}
\begin{proof}
    Triviale Induktion
\end{proof}
\begin{Bemerkung}
    Es wird sich später erweisen, dass wir für bestimmte Konstruktionen von austauschbaren Partitionen von $\mathbb{N}$ die \textit{EPPF} explizit nachprüfen sollen. Dafür ist folgendes Lemma nützlich.
\end{Bemerkung}
\begin{lemma}
    \label{Austauschbar Lemma}
    Sei $N_m$, $1 \leq m \leq n$ ein zufälliges Element in der Größe der Klassen einer Partition $\Pi_m$ in Reihenfolge ihres Erscheinens. Falls $(N_1,...,N_n)$ eine Markov Kette ist mit 
    \[
    \mathbb{P}(N_{m+1} = \textbf{n}^{i+} | N_{m} = \textbf{n}) = \frac{p(\textbf{n}^{i+})}{p(\textbf{n})} 
    \]
   für ein $p: \bigcup_{i = 1}^{n}\mathbb{N}^i \rightarrow [0,1]$,$p(1) = 1$, dann ist  
    $\Pi_n$ partiell Austauschbar mit \textit{PEPF} $p$.
\end{lemma}
\begin{proof}
    
\end{proof}

\end{section}