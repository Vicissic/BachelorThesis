\begin{section}{Motivation}
% LTeX: language=de-DE
Als vorläufige Motivation für die Relevanz der Theorie der austauschbaren und partiell austauschbaren Partitionen von $\mathbb{N}$ für die darauffolgende Baumanalyse, beschreiben wir kurz informell den Zusammenhang. Die Konstruktion von Bäumen die betrachtet werden, beinhalten sukzessives Hinzufügen von Knoten, welche mit natürlichen Zahlen identifiziert werden. Teilbäume, wo die Knoten/natürliche Zahlen im Baum angefügt werden, kann man sich deshalb als eine Art Partition vorstellen. Die Austauschbarkeit erlaubt uns eine De-Finetti-ähnlichen Aussage (Satz \ref{Main theorem CRP}) über die relative Größe von Partitionen und somit über die Größe von Teilbäumen zu formulieren. \\
Wir beginnen mit den Definitionen. 
\end{section}
\begin{section}{Definitionen}
\begin{Definition}
   Sei $F$ eine endliche Menge. Dann ist eine Partition von $F$ eine ungeordnete Sammlung von nichtleeren, disjunkten Untermengen $\{A_i| 1 \leq i \leq k, A_i \subseteq F\}$, mit $\cup_{i=1}^{k}A_k = F$.\\
   Eine Partition von einer natürlichen Zahl $n$, ist eine ungeordnete Menge von positiven natürlichen Zahlen mit Summe $n$.
\end{Definition}
\begin{Definition}
    \textnormal{(Zufällige Partitionen)}
\begin{enumerate}
    \item Eine zufällige Partition von $n\in \mathbb{N}$ ist eine Zufallsvariable $\sigma_n$ mit Werten in der endlichen Menge von allen Partitionen von $n$.
    \item Eine zufällige Partition von $[n] := \{1 ...n\}$, ist eine Zufallsvariable $\Pi_n$ mit Werten in $\mathcal{P}(n)$, der endlichen Menge von Partitionen von $[n]$.
    \item Eine zufällige Partition auf $\mathbb{N}$ ist eine Folge $\Pi = (\Pi_n)_{n\in \mathbb{N}}$ von Zufallsvariablen auf einem gemeinsamen Wahrscheinlichkeitsraum mit Werten in der endlichen Menge von Partitionen von $[n]$, sodass für $i<j, i,j \in \mathbb{N}$ die Restriktion von $\Pi_j$ auf $[i]$ $\Pi_i$ ist.
\end{enumerate}

\end{Definition}
\begin{Definition}
    \textnormal{(Austauschbarkeit)}
    \label{Austauschbarkeit}
    \begin{enumerate}
        \item Eine zufällige Partition $\Pi_n$ von $[n]$ heißt partiell Austauschbar, falls für jede Partition $\{ A_1 ... A_k\}$ von $[n]$ in Reihenfolge ihres Erscheinens\footnote{\textit{In Reihenfolge ihres Erscheinens} bedeutet $1 \in A_1$, und für alle $2 \leq i \leq k$, das erste Element von $[n]\setminus \cup_{j=1}^{i-1}A_j$ zu $A_i$ gehört}
        \[ 
            \mathbb{P}(\Pi_n = \{ A_1 ... A_k\}) = p(|A_1|, ... |A_k|)
        \]  
        für eine Funktion p von Kompositionen $\textbf{n}:= (n_1, ..., n_k)$ von $n$ gilt. Diese Funktion wird auch als PEPF (Partially Exchangeable Probability Function) bezeichnet.
        \item Falls die Funktion $p$ zusätzlich noch symmetrisch ist, so ist die zufällige Partition $\Pi_n$ Austauschbar und man bezeichnet $p$ als \textit{EPPF} (Exchangeable Partition Probability Function).
    \end{enumerate}
\end{Definition}
\begin{Bemerkung}
    Dass bei einer symmetrischen Funktion $p$, die zufällige Partition tatsächlich Austauschbar ist, folgt aus der Definition und der Diskussion in \cite[Seite 85]{aldous2006ecole}: eine Partition heißt austauschbar, falls 
    \[
         \mathcal{L}(\Pi_n) = \mathcal{L}(\pi(\Pi_{n})),
    \]
    für jede Permutation $\pi$ von $\{1...n\}$ gilt. Hier wirkt $\pi$ auf Partitionen $(A_1,...A_k)$ von $n$ über $\pi(A_1,...,A_k) =(\pi(A_1),...,\pi(A_k)) $ und $\pi(A_i) = \{\pi(i) | i \in A_i\}$.
\end{Bemerkung}
\begin{Definition}{Konsistenz}\\
    Wir nennen eine Folge von austauschbaren zufälligen Partitionen $(\Pi_n)$ \textit{konsistent}, falls die Restriktion $\Pi_{m,n}$ von $\Pi_m$ zu $[n]$ für alle $m > n$ die gleiche Verteilung hat wie $\Pi_m$. Alle zufälligen Partitionen auf $\mathbb{N}$ sind konsistent.
\end{Definition}
\begin{lemma}
    \label{lemma partielle austauschbarkeit}
    Sei $\Pi_n$ eine partiell austauschbare zufällige Partition von $[n]$ mit \textit{PEPF} $p(\textbf{n})$ und $\Pi_m$ die Restriktion von $\Pi_n$ auf $[m], m \in \{1...n\}$. Dann gilt mit 
    \[
    \textbf{m} := (m_1,...,m_k), \hspace*{5pt} \sum_{i=1}^{k}m_i = m,\hspace*{5pt} m_i \in \mathbb{N} 
    \]
    und 
    \[ 
    \textbf{m}^{i+} = \textbf{m} + e_i =\textbf{m} + \overbrace{\underbrace{(0,...,0}_{i-1 \textnormal{ 0-en}}1,0,...,0)}^{k+1} , i \in \{1...k+1\}
    \]
     für Einheitsvektoren $e_i$:
\begin{center} 
    $\Pi_m$ ist partiell Austauschbar mit PEPF $p(\textbf{m}):=  \sum_{j=1}^{k+1} p(\textbf{m}^{j+})$.
\end{center}
\end{lemma}
\begin{proof}
    Leichte Induktion, siehe \cite[Proposition 10]{pitman1995exchangeable}
\end{proof}
\begin{Bemerkung}
    Es wird sich später erweisen, dass wir für bestimmte Konstruktionen von austauschbaren Partitionen von $\mathbb{N}$ die \textit{EPPF} explizit nachprüfen sollen. Dafür ist folgendes Lemma nützlich.
\end{Bemerkung}
\begin{lemma}
    \label{Austauschbar Lemma}
    Sei $N_m$, $1 \leq m \leq n$ ein zufälliges Element in der Größe der Klassen einer Partition $\Pi_m$ in Reihenfolge ihres Erscheinens. Falls $(N_1,...,N_n)$ eine Markov Kette ist mit 
    \begin{equation}
    \label{markov chain}
    \mathbb{P}(N_{m+1} = \textbf{n}^{i+} | N_{m} = \textbf{n}) = \frac{p(\textbf{n}^{i+})}{p(\textbf{n})} 
    \end{equation}
   für ein $p: \bigcup_{i = 1}^{n}\mathbb{N}^i \rightarrow [0,1]$,$p(1) = 1$, dann ist $\Pi_n$ partiell Austauschbar mit \textit{PEPF} $p$.
\end{lemma}
\begin{proof}
    Wir bemerken zuerst aus Definition \ref{Austauschbarkeit}, falls wir $N^*_n:= \{\textbf{n}:= (n_1,...,n_k)|k \in \mathbb{N},\sum_{i=1}^{k}n_i = n\}$ definieren, dass $p$ eine \textit{PEPF} ist, falls $S_n := \sum_{\textbf{n} \in N^*_n}\#(\textbf{n})p(\textbf{n}) = 1$ und $p(\textbf{n}) \in [0,1]$. Hier ist 
    \begin{equation}
        \label{combinatorial equation}
        \#(\textbf{n}):= \frac{n!}{n_k(n_k + n_{k-1})...(n_k+...+n_1)\prod_{i=1}^{k}(n_i-1)!},
    \end{equation}
    die Anzahl der Partitionen von $[n]$ mit Klassengrößen in Reihenfolge ihres Erscheinens durch $\textbf{n}$ gegeben (Siehe Bemerkung \ref{combinatorial bemerkung}). Angenommen $S_n \neq 1$, dann gilt aber $S_n> 0$ und wir definieren $q(\textbf{n}) := \frac{p(\textbf{n})}{S_n}$. Dann ist $q(\textbf{n})$ aus Lemma \ref{lemma partielle austauschbarkeit} zusammen mit \ref{markov chain} eine \textit{PEPF}, also $q(1) = \frac{p(1)}{S_n} = 1 \Rightarrow S_n =1$, somit ist $p(\textbf{n})$ eine \textit{PEPF} und die Behauptung ist gezeigt. 
\end{proof}
\begin{Bemerkung}
    \label{combinatorial bemerkung}
    Das $\#(\textbf{n})$ tatsächlich die Form \ref{combinatorial equation} hat, wird im Anhang, Lemma \ref{combinatorial lemma}, bewiesen und ist für die restliche Ausarbeitung nicht weiter relevant.
\end{Bemerkung}
\end{section}