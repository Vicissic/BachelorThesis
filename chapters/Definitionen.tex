\begin{section}{Definitionen}
% LTeX: language=de-DE
\begin{Definition}
   Sei $F$ eine endliche Menge. Dann ist eine Partition von $F$ eine ungeordnete Sammlung von nichtleeren, disjunkten Untermengen $\{A_i| 1 \leq i \leq k, A_i \subseteq F\}$, mit $\cup_{i=1}^{k}A_k = F$.\\
   Eine Partition von einer natürlichen Zahl $n$, ist eine ungeordnete Menge von positiven natürlichen Zahlen mit Summe $n$.
\end{Definition}
\begin{Definition}
    \textnormal{(Zufällige Partitionen)}
\begin{enumerate}
    \item Eine zufällige Partition von $n\in \mathbb{N}$ ist eine Zufallsvariable $\pi_n$ mit Werten in der endlichen Menge von allen Partitionen von $n$.
    \item Eine zufällige Partition von $[n] := \{1 ...n\}$, ist eine Zufallsvariable $\Pi_n$ mit Werten in der endlichen Menge von Partitionen von $[n]$.
    \item Eine zufällige Partition auf $\mathbb{N}$ ist eine Folge $\Pi = (\Pi_n)_{n\in \mathbb{N}}$ von Zufallsvariablen auf einem gemeinsamen Wahrscheinlichkeitsraum mit Werten in der endlichen Menge von Partitionen von $[n]$, sodass für $i<j, i,j \in \mathbb{N}$ die Restriktion von $\Pi_j$ auf $[i]$ $\Pi_i$ ist.
\end{enumerate}

\end{Definition}
\begin{Definition}
    \textnormal{(Austauschbarkeit)}
    \begin{enumerate}
        \item Eine zufällige Partition $\Pi_n$ von $[n]$ heißt Austauschbar, falls für jede Partition $\{ A_1 ... A_n\}$ von $[n]$
        \[ 
            \mathbb{P}(\Pi_n = \{ A_1 ... A_k\}) = p(|A_1|, ... |A_k|)
        \]  
        für eine symmetrische Funktion p von Kompositionen $(n_1, ..., n_k)$ von $n$. Diese Funktion wird auch EPPF (Exchangeable partition probability function) bezeichnet.

    \end{enumerate}
\end{Definition}

\end{section}