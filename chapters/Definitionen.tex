\begin{section}{Definitionen}
% LTeX: language=de-DE
\begin{Definition}
   Sei $F$ eine endliche Menge. Dann ist eine Partition von $F$ eine ungeordnete Sammlung von nichtleeren, disjunkten Untermengen $\{A_i| 1 \leq i \leq k, A_i \subseteq F\}$, mit $\cup_{i=1}^{k}A_k = F$.\\
   Eine Partition von einer natürlichen Zahl $n$, ist eine ungeordnete Menge von positiven natürlichen Zahlen mit Summe $n$.
\end{Definition}
\begin{Definition}
    \textnormal{(Zufällige Partitionen)}
\begin{enumerate}
    \item Eine zufällige Partition von $n\in \mathbb{N}$ ist eine Zufallsvariable $\pi_n$ mit Werten in der endlichen Menge von allen Partitionen von $n$.
    \item Eine zufällige Partition von $[n] := \{1 ...n\}$, ist eine Zufallsvariable $\Pi_n$ mit Werten in $\mathcal{P}(n)$, der endlichen Menge von Partitionen von $[n]$.
    \item Eine zufällige Partition auf $\mathbb{N}$ ist eine Folge $\Pi = (\Pi_n)_{n\in \mathbb{N}}$ von Zufallsvariablen auf einem gemeinsamen Wahrscheinlichkeitsraum mit Werten in der endlichen Menge von Partitionen von $[n]$, sodass für $i<j, i,j \in \mathbb{N}$ die Restriktion von $\Pi_j$ auf $[i]$ $\Pi_i$ ist.
\end{enumerate}

\end{Definition}
\begin{Definition}
    \textnormal{(Austauschbarkeit)}
    \label{Austauschbarkeit}
    \begin{enumerate}
        \item Eine zufällige Partition $\Pi_n$ von $[n]$ heißt partiell Austauschbar, falls für jede Partition $\{ A_1 ... A_k\}$ von $[n]$ in Reihenfolge ihres Erscheinens\footnote{\textit{In Reihenfolge ihres Erscheinens} bedeutet $1 \in A_1$, und für alle $2 \leq i \leq k$, das erste Element von $[n]\setminus \cup_{j=1}^{i-1}A_j$ zu $A_i$ gehört}
        \[ 
            \mathbb{P}(\Pi_n = \{ A_1 ... A_k\}) = p(|A_1|, ... |A_k|)
        \]  
        für eine Funktion p von Kompositionen $\textbf{n}:= (n_1, ..., n_k)$ von $n$ gilt. Diese Funktion wird auch als PEPF (Partially Exchangeable Probability Function) bezeichnet.
        \item Falls die Funktion $p$ zusätzlich noch symmetrisch ist, so ist die zufällige Partition $\Pi_n$ Austauschbar und man bezeichnet $p$ als \textit{EPPF} (Exchangeable Partition Probability Function).
    \end{enumerate}
\end{Definition}
\begin{Bemerkung}
    Das bei einer symmetrischen Funktion $p$, die zufällige Partition tatsächlich Austauschbar ist, folgt aus der Definition und der Diskussion in \cite[Seite 85]{aldous2006ecole}
\end{Bemerkung}
\begin{Definition}{Konsistenz}\\
    Wir nennen eine Folge von austauschbaren zufälligen Partitionen $(\Pi_n)$ \textit{konsistent}, falls die Restriktion $\Pi_{m,n}$ von $\Pi_m$ zu $[n]$ für alle $m > n$ die gleiche Verteilung hat wie $\Pi_m$. Alle zufälligen Partitionen auf $\mathbb{N}$ sind konsistent.
\end{Definition}
\begin{lemma}
    \label{lemma partielle austauschbarkeit}
    Sei $\Pi_n$ eine partiell austauschbare zufällige Partition von $[n]$ mit \textit{PEPF} $p(\textbf{n})$ und $\Pi_m$ die Restriktion von $\Pi_n$ auf $[m], m \in \{1...n\}$, dann gilt mit $\textbf{m} := (m_1,...,m_k)\in \bigcup_{i = 1 }^{n} \mathbb{N}^i$, $\sum_{i=1}^{k}m_i = m$ und $\textbf{m}^{i+} = \textbf{m} + e_i, i \in \{1...k+1\}$ für einen Einheitsvektor $e_i$:
\begin{center} 
    $\Pi_m$ ist partiell Austauschbar mit PEPF $p(\textbf{m}):=  \sum_{j=1}^{k+1} p(\textbf{m}^{j+})$.
\end{center}
\end{lemma}
\begin{proof}
    Leichte Induktion, siehe \cite[Proposition 10]{pitman1995exchangeable}
\end{proof}
\begin{Bemerkung}
    Es wird sich später erweisen, dass wir für bestimmte Konstruktionen von austauschbaren Partitionen von $\mathbb{N}$ die \textit{EPPF} explizit nachprüfen sollen. Dafür ist folgendes Lemma nützlich.
\end{Bemerkung}
\begin{lemma}
    \label{Austauschbar Lemma}
    Sei $N_m$, $1 \leq m \leq n$ ein zufälliges Element in der Größe der Klassen einer Partition $\Pi_m$ in Reihenfolge ihres Erscheinens. Falls $(N_1,...,N_n)$ eine Markov Kette ist mit 
    \begin{equation}
    \label{markov chain}
    \mathbb{P}(N_{m+1} = \textbf{n}^{i+} | N_{m} = \textbf{n}) = \frac{p(\textbf{n}^{i+})}{p(\textbf{n})} 
    \end{equation}
   für ein $p: \bigcup_{i = 1}^{n}\mathbb{N}^i \rightarrow [0,1]$,$p(1) = 1$, dann ist $\Pi_n$ partiell Austauschbar mit \textit{PEPF} $p$.
\end{lemma}
\begin{proof}
    Wir bemerken zuerst aus Definition \ref{Austauschbarkeit}, falls wir $N^*_n:= \{\textbf{n}:= (n_1,...,n_k)|k \in \mathbb{N},\sum_{i=1}^{k}n_i = n\}$ definieren, dass $p$ eine \textit{PEPF} ist, falls $S_n := \sum_{\textbf{n} \in N^*_n}\#(\textbf{n})p(\textbf{n}) = 1$ und $p(\textbf{n}) \in [0,1]$. Hier ist 
    \begin{equation}
        \label{combinatorial equation}
        \#(\textbf{n}):= \frac{n!}{n_k(n_k + n_{k-1})...(n_k+...+n_1)\prod_{i=1}^{k}(n_i-1)!},
    \end{equation}
    die Anzahl der Partitionen von $[n]$ mit Klassengrößen in Reihenfolge ihres Erscheinens durch $\textbf{n}$ gegeben (Siehe Bemerkung \ref{combinatorial argument}). Angenommen $S_n \neq 1$, dann gilt aber $S_n> 0$ und wir definieren $q(\textbf{n}) := \frac{p(\textbf{n})}{S_n}$. Dann ist $q(\textbf{n})$ aus Lemma \ref{lemma partielle austauschbarkeit} zusammen mit \ref{markov chain} eine \textit{PEPF}, also $q(1) = \frac{p(1)}{S_n} = 1 \Rightarrow S_n =1$, somit ist $p(\textbf{n})$ eine \textit{EPPF} und die Behauptung ist gezeigt. 
\end{proof}
\begin{Bemerkung}
    \label{combinatorial argument}
    Das $\#(\textbf{n})$ tatsächlich die Form \ref{combinatorial equation} hat, lässt sich durch Betrachtung von
    \begin{align*}
   \Omega_{\textbf{n}} = \{ \textbf{x}:= (x_{n_1}^{1},x_{n_1}^{2},...,x_{n_1}^{n_1},x_{n_2}^{1},x_{n_2}^{2},...,x_{n_k}^{n_k})|& \textbf{n}:= (n_1,...,n_k), \sum_{i=1}^{k}n_i=  n \\
   &,\textbf{x} \in \mathcal{P}(n)\\
   &, x_{n_i}^j < x_{n_i}^{j+1}\hspace{5pt} \forall  i \in \{1...k\} \forall j \in \{1...n_i-1\} \\
   &, x^1_{n_i} < x^1_{n_i+1} \hspace{5pt} \forall i \in \{1...k\}\}
    \end{align*} 
    nachweisen. Es gilt
    \[
   \#(\textbf{n}) =  |\Omega_\textbf{n}|
    \]
    und wir berechnen nun $|\Omega_\textbf{n}|$. Es gibt $n!$ Möglichkeiten $[n]$ zu permutieren, nach den Bedingungen von $\Omega_n$ muss aber $x_{n_i}^1$ für alle $i$ kleiner sein als jede darauffolgende Zahl in der Liste. Formal beschrieben bedeutet das, falls $\textbf{x}_i, 1 \leq i \leq n$ die i-te Komponente von $\textbf{x}$ ist, so ist $\textbf{x}_{n_i} < \textbf{x}_{j}$ $\forall i \in \{n_i+1,...,n\}$. Sei  
    \begin{align*}
   \Omega^1_{\textbf{n}} = \{ \textbf{x}:= (x_{n_1}^{1},x_{n_1}^{2},...,x_{n_1}^{n_1},x_{n_2}^{1},x_{n_2}^{2},...,x_{n_k}^{n_k})|& \textbf{n}:= (n_1,...,n_k), \sum_{i=1}^{k}n_i=  n \\
   &,\textbf{x} \in \mathcal{P}(n)\\
   &, \textbf{x}_{n_i} < \textbf{x}_j \forall i \in \{1...k\},\forall j \in \{n_i+1,...,n\}\}
    \end{align*} 
    so ist
    \[
    |\Omega^1_{\textbf{n}}| = \frac{n!}{n(n-n_1)(n-n_2-n_1)...(n-\sum_{i=1}^{k-1}n_i)}.
    \]
    Um von $|\Omega_\textbf{n}^1|$ auf $|\Omega_\textbf{n}|$ zu kommen fehlt uns in jedem der einzelnen Komponenten $n_i$ die Permutationen der $n_i -1$ nachfolgenden Zahlen nach jedem der $\textbf{x}_{n_i}$ zu zählen. Da es aber $\prod_{i=1}^{k}(n_i-1)!$ davon gibt, gilt 
    \[
     |\Omega_\textbf{n}| = \frac{|\Omega_\textbf{n}^1|}{\prod_{i=1}^{k}(n_i-1)!}
    \]
    und die Behauptung ist gezeigt.
\end{Bemerkung}
\end{section}