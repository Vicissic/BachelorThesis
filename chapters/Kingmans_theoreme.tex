\begin{section}{Kingman Paintbox representation}
% LTeX: language=de-DE
Oftmals ist es nützlich, die Menge $\{|A_1|,|A_2|,...,|A_k|\}$ der Größe der Blöcke einer Partition $P_n$ von $[n]$ mit einer Folge von nicht negativen ganzen Zahlen zu identifizieren. Es gibt verschiedene Möglichkeiten dies zu machen \cite[Seite 15]{pitman2006combinatorial}. Für unsere Zwecke reicht es einer dieser Repräsentationen zu betrachten.
\\
Für eine Partition $P_n$ definieren wir $|P_n|$ als die Anzahl der Blöcke und betrachten die absteigende Folge der Größen der Blöcke $(N^\downarrow_{n,1},N^\downarrow_{n,2},...,N^\downarrow_{n,|P_n|},0,0...)$. Wir fügen also nach dem kleinsten Block $N^\downarrow_{n,|P_n|}$, eine unendliche Folge von Nullen hinzu. Wir haben also eine bijektive Identifizierung von der Menge der Partitionen von $n$ zur folgenden Menge der Nichtnegativen natürlichen Zahlen:
\[
 \{(n_j)_{1\leq j \leq \infty}: n_1 \geq n_2 \geq ... \geq 0 \text{ und } \sum_{j=1}^{\infty}{n_j} = n\}
\] 

Das Ziel ist es nun, eine ähnliche Aussage wie im Satz von de Finetti für Partitionen von $\mathbb{N}$ zu formulieren. Dazu starten wir mit dem Satz selbst.
\begin{theorem}\cite[Theorem 3.1]{aldous2006ecole}{\textbf{De Finetti}}\\
    Eine austauschbare Folge $(X_i)_{i \in \mathbb{N}}$ ist eine Mischung von i.i.d Folgen.
\end{theorem}
\begin{Bemerkung}
    Für die Zwecke dieser Arbeit reicht es \textit{eine Mischung von i.i.d Folgen}, als \glqq es existiert ein zufälliges W-Maß $\mu$, sodass bedingt auf $\mu$, die $X_i$ i.i.d sind\grqq, zu interpretieren. Auf die formale Definition verweisen wir auf \cite[Abschnitt 1.2]{aldous2006ecole}.
\end{Bemerkung}
\begin{Bemerkung}
    Die Äquivalenz dieser Formulierung von de Finetti zu der Formulierung \textit{i.i.d gegeben einer $\sigma$-Algebra} wird in \cite[Lemma 2.18]{aldous2006ecole} bewiesen.  
\end{Bemerkung}
Sei $\mu$ eine zufällige Verteilung mit Werten in $[0,1]$ und $(X_i)_{i \in \mathbb{N}}$ eine Folge von unabhängig identisch verteilten Zufallsvariablen mit Verteilung $\mu$. Sei $\Pi(\omega)$ die zufällige Partition mit Komponenten $\{i| X_i(\omega) = x\},  0 \leq x \leq 1$. Die Bezeichnung von diesen austauschbaren\footnote{Die Austauschbarkeit folgt aus Betrachtung von $\Pi_{i,j} := \{\omega | i \text{ und } j \text{ in derselben Komponente von } \Pi(\omega)\}$ und aus der Definition der Austauschbarkeit: $\Pi$ ist austauschbar genau dann, wenn $\Pi_{i,j} \stackrel{\mathcal{L}}{=} \Pi_{\pi(i),\pi(j)} \text{für alle Permutationen }\pi_n \text{auf }[n] \text{ gilt}$} Prozess als \textit{Paintbox Prozess} stammt aus der Vorstellung, dass man Objekte $\{1,2 ...\}$ mit Farben $X_i$ markiert (Die Zahlen $x \in [0,1]$ interpretiert man als Farben-Spektrum). 
Man erhält somit eine Partition von Mengen von gleichfarbigen Objekten. Die Verteilung der zufälligen Partition $\Pi$ in dieser Konstruktion hängt nur von den Größen der Atome von $\mu$ ab ($\{x \in [0,1] | \mu(x) > 0 \}$). Wir definieren $p^\downarrow_j := \mu(x_j)$, sodass die $x_j$ die Atome von $\mu$ sind und $p_j^\downarrow$ eine absteigende Folge ist. Die Aussage von \textit{Kingman's representation} ist, dass bereits jede austauschbare Folge von $\mathbb{N}$ so konstruiert werden kann, also nur von der Folge $(p^\downarrow_i)_{i \in \mathbb{N}}$ abhängig ist. Um das Formal aufzufassen, definieren wir:
\begin{align*}
L: \mathcal{M}_1([0,1]) &\to \nabla \\
\mu \hspace{20pt} &\to (p_j^\downarrow)_{j \in \mathbb{N}},
\end{align*}
und nennen den oben beschrieben Prozess zur Verteilung $\mu$ den $L(\mu) = (p_j^\downarrow)_{j \in \mathbb{N}}$ \textit{-Paintbox Prozess}. Der Satz besagt nun:
\begin{theorem}
\textnormal{Kingman's Representation}\\
    Sei $\Pi := (\Pi_n)_{n \in \mathbb{N}}$ eine austauschbare zufällige Partition von $\mathbb{N}$ und $(N^{\downarrow}_{n,i}, i \geq 1)$ die fallende Folge der Blockgrößen von $\Pi_n$, mit $N_{n,i}^\downarrow = 0$, falls $\Pi_n$ weniger als $i$ Blöcke hat. Dann konvergiert $\frac{N_{n,i}^\downarrow}{n} $ $ \forall i \geq 1$ P-f.s. für $n \to \infty$ gegen ein $P^\downarrow_i$. Weiter ist die bedingte Verteilung von $\Pi_\infty$ gegeben $(P_i^\downarrow, i \geq 1)$ gleich der Verteilung von einem $(P_i^\downarrow, i \geq 1)$ \textit{Paintbox Prozess}.
\end{theorem}
\begin{proof}
Sei $(U_i)_{i \in \mathbb{N}}$ eine Folge unabhängig in $[0,1]$ uniform verteilter Zufallsvariablen. Wir definieren 
\begin{align*}
F_i(\omega) &= \min{\{j| i \text{ und } j \text{ sind in der gleichen Komponente von }  \Pi(\omega)\}} \leq i \\
Z_i &= U_{F_i}.
\end{align*}
$\Pi(\omega)$ ist also eine Partition mit Komponenten $\{i| Z_i(\omega) = z\}$ für $0 \leq z \leq 1$. $(Z_i)_{i \in \mathbb{N}}$ ist austauschbar, da wir $(Z_i)_{i \in \mathbb{N}} = g((\Pi_i)_{i \in \mathbb{N}},(U_i)_{i \in \mathbb{N}})$ für eine Abbildung $g$ auffassen können, und da Abbildungen austauschbarer Folgen austauschbar sind\footnote{Man beachte, dass hier die Folge $U_i$ auch von $\Pi$ unabhgängig sein sollte.}. Nach De Finetti gibt es ein zufälliges Maß $\alpha$, sodass bedingt auf $\alpha = \mu$, die $Z_i$ unabhängig identisch wie $\mu$ verteilt sind. Dies entspricht aber genau dem vorher beschriebenen $L(\mu)$\textit{-Paintbox Prozess} und die zweite Behauptung ist gezeigt. Für die erste Behauptung nutzen wir das starke Gesetz großer Zahlen. Dafür sei $\phi$ ein beliebiges Maß mit $L(\phi) = (p^\downarrow_i)_{i \in \mathbb{N}}$ und $X_i$ i.i.d $\phi$ verteilt. Dann gilt nach dem Gesetz $\frac{1}{n}\sum_{i=1}^{n}\chi_{\{X_i = x\}} \rightarrow \mathbb{P}(X_i = x) = \phi(x)$ P-f.s. für $n \to \infty$. Somit folgt auch $L_n(\Pi_n)...$ und wir sind fertig.\textcolor{red}{finish this proof and adjust this whole chapter.}
\end{proof}

\end{section}