\chapter{Anwendungen}
% LTeX: language=de-DE
\begin{Definition}
    \label{definition height trees}
    Für einen gewurzelten Baum $T$, sei $h(v)$ die Tiefe eines Knotens $v$ und sei $v\wedge w$ der letzte gemeinsame Vorfahren von $v$ und $w$. Dann definieren wir die Funktion $Y$ als:
    \[
        Y := Y(T) = \sum_{v \neq w}h(v \wedge w),
    \]
    wobei wir die Summe über geordnete Paare $(v,w)$ definieren. Die Summe über geordnete Paare ist durch Symmetrie von $v \wedge w$ gleich dem doppelten der Summe über ungeordnete Paare. 
\end{Definition}
\begin{theorem}\textnormal{Hoeffdings Gesetz großer Zahlen für U-Statistiken}\\
    \label{hoeffding theorem}
    Seien $X_i$ i.i.d Zufallsvariablen, $f(X_1,...X_r)$ eine Funktion mit Erwartungswert $\theta$ und sei $\tilde{f}_n$ definiert durch 
    \[
        \frac{1}{(n)_{k}}\sum_{\substack{i_1,...,i_k \in \{1...n\}\\ \textnormal{paarweise verschieden}}}f(X_{i_1},...,X_{i_k}),
    \]
    wobei $(n)_k := \prod_{i=0}^{k-1}(n-i)$.
    Dann konvergiert 
    \[
        \tilde{f}_n  \rightarrow \theta \hspace{5pt} \textnormal{P-f.s.} \hspace{5pt} \textnormal{für } n \rightarrow \infty.
    \]
\begin{proof}
    Das Gesetz wird in \cite{hoeffding1961strong} bewiesen.
\end{proof}
\end{theorem}
\begin{theorem}
    \label{big Q theorem}
    Sei $(T_n^\mathcal{P})$ ein zufälliger Split Baum für einen \textit{Split Vektor} $\mathcal{P} = (P_i)_{i \in \mathbb{N}}$ und sei $Y_n := Y(T^\mathcal{P}_n)$ definiert wie in \ref{definition height trees}. Wir nehmen an, dass für ein $i \in \mathbb{N}$, $0< P_i < 1$ mit positiver Wahrscheinlichkeit. Dann existiert eine Zufallsvariable $Q$, sodass $Y_n/n^2 \rightarrow Q$ P-f.s. für $n \to \infty$. Weiter gilt 
    \begin{equation}
        \label{first equation of Q theorem}
    \mathbb{E}[Q]  = \frac{1}{1-\mathbb{E}[\sum_{i \geq 1}P_i ]}-1 < \infty. 
    \end{equation} 
    Zusätzlich erfüllt $Q$ die Verteilungs-Fixpunktgleichung (Distributional Fixed point equation)
    \begin{equation}
        Q \stackrel{\mathcal{L}}{=} \sum_{i=1}^{\infty}P_i^2(1+Q^{(i)}),
        \label{second equation of Q theorem}
    \end{equation} 
    wobei die $Q^{(i)}$ unabhängig voneinander und von $(P_i)_{i \in \mathbb{N}}$ sind, und $Q^{(i)} \stackrel{\mathcal{L}}{=}Q$. 
    Weiter gilt, mit $W$ so definiert wie nach Bemerkung \ref{Bemerkung PD Verteilungen}, dass 
\begin{equation}
    \mathbb{E}[Q] = \frac{\mathbb{E}[W]}{1 - \mathbb{E}[W]}.
        \label{third equation of Q theorem}
\end{equation}
\end{theorem}
\begin{proof}
  Wir beginnen mit einer Approximation von $Y_n$ für Split Bäume. Dafür modifizieren wir die Konstruktion von Split Bäumen und nehmen an, dass Anfangs bereits jeder Knoten \textit{voll} ist. Bedeutet, dass bereits der erste Knoten nicht an der Wurzel hängen bleibt, sondern sich anhand der jeweiligen \textit{Split Vektoren} der Knoten unendlich Tief entlang des Baums bewegt. Wir notieren den Pfad vom k-ten Ball, der sich so entlang des Baumes bewegt mit $\textbf{X}_k = (X_{k,i})_{i \in \mathbb{N}}$, wobei $X_{k,i} \in \mathbb{N}$ die Knotenbeschriftung vom Pfad in der  Tiefe $i$ bezeichnet. Gegeben \textit{Split Vektoren} $\mathcal{P}^{v}$ von allen Knoten $v$, sind die $\textbf{X}_i$ i.i.d. durch die Verteilung 
  \begin{equation}
    \label{independency sequence depth}
    \mathbb{P}(\bigcap_{j = 1}^{m}X_{k,j} = i_j) = \prod_{j=1}^{m}P_{i_j}^{(i_1...i_{j-1})}
  \end{equation}
  festgelegt, wobei wir mit $(i_1...i_{j-1})$ den Knoten in der $(j-1)$-ten Tiefe vom Pfad identifizieren. Sei nun für 2 Sequenzen $\textbf{X}^1,\textbf{X}^2 \in \mathbb{N}^{\infty}$ die Funktion $f$ folgendermaßen definiert:
  \[
    f(\textbf{X}^1,\textbf{X}^2) = min(\{i: \textbf{X}^1_i \neq \textbf{X}^2_i\}) - 1,
  \]
  also die Länge der längsten gemeinsamen Anfangssequenz. Dann ist 
  \[
    f(\textbf{X}^k,\textbf{X}^l) = h(v_k,v_l),
  \]
  falls beide gegenseitig kein Vorfahren vom anderen sind und $\textbf{X}^k$ bzw. $\textbf{X}^l$ die zugehörende Sequenz vom Knoten $v_k$ bzw. $v_l$ ist. In dem Falle das $v_l$ ein Vorfahren von $v_k$ ist, gilt 
  \begin{equation}
    \label{inequality for height}
    f(\textbf{X}^k,\textbf{X}^l) \geq h(v_k,v_l).
  \end{equation} \\
  Mithilfe dieser Definition definieren wir unsere Approximation durch 
  \[
    \tilde{Y}_n = 2\sum_{l < k \leq n}f(\textbf{X}^k,\textbf{X}^l).
  \]
Bedingt auf alle \textit{Split Vektoren} $\mathcal{P}^{v}$, bekommen wir mit $\chi$ definiert als Indikator Funktion. \textcolor{red}{Warum verschwindet die Bedingung?}
\begin{align}
\mathbb{E}[f(\textbf{X}^1,\textbf{X}^2)|\{\mathcal{P}^{v}, v \textnormal{ Knoten}\}] &=\sum_{m=1}^{\infty}\mathbb{P}({f(\textbf{X}^1,\textbf{X}^2) \geq m}|\{\mathcal{P}^{v}, v \textnormal{ Knoten}\}) \nonumber\\
&=\sum_{m=1}^{\infty}\mathbb{P}(\bigcap_{j=1}^{m}\textbf{X}_{1,j}=\textbf{X}_{2,j})   \nonumber\\
&=\sum_{m=1}^{\infty}\mathbb{P}(\bigcup_{i_1,...,i_m \in \mathbb{N}} \{\bigcap_{j=1}^{m}\textbf{X}_{1,j}=\textbf{X}_{2,j} = i_j) \}   \nonumber\\
&=\sum_{m=1}^{\infty}\sum_{i_1,...,i_m \in \mathbb{N}}\mathbb{P}(\{\bigcap_{j=1}^{m}\textbf{X}_{1,j}=\textbf{X}_{2,j} = i_j \}) \nonumber\\
&=\sum_{m=1}^{\infty}\sum_{i_1,...,i_m \in \mathbb{N}}\mathbb{P}(\bigcap_{j=1}^{m}\textbf{X}_{1,j} = i_j )\mathbb{P}(\bigcap_{j=1}^{m}\textbf{X}_{2,j} = i_j ) \nonumber\\
&=\sum_{m=1}^{\infty}\sum_{i_1,...,i_m \in \mathbb{N}}(\prod_{j=1}^{m}P_{i_j}^{(i_1...i_{j-1})})^2 =: Q \label{Gleichungskette Tiefe}.
\end{align}
Mit der Turmeigenschaft zusammen mit der Voraussetzung, dass die $P^{(v)}$ i.i.d sind, folgt 
\begin{align}
\mathbb{E}[f(\textbf{X}^1,\textbf{X}^2) ] = \mathbb{E}[Q] &=\sum_{m=1}^{\infty}\sum_{i_1,...,i_m \in \mathbb{N}}\prod_{j=1}^{m}\mathbb{E}[P_{i_j}^{2}] \nonumber\\
 &= \sum_{m=1}^{\infty}(\sum_{i =1}^{\infty}\mathbb{E}[P_{i}^{2}])^m \nonumber \\
 &= \frac{1}{1- \sum_{i=1}^{\infty}\mathbb{E}[P_i^2]}-1  = \frac{\sum_{i=1}^{\infty}\mathbb{E}[P_i^2]}{1- \sum_{i=1}^{\infty}\mathbb{E}[P_i^2]}  \nonumber.
\end{align}
Wobei wir in die letzte Gleichung die Summenformel für eine geometrische Reihe ausgenutzt haben, da $\sum_{i=1}^{\infty} P_i^2 \leq \sum_{i=1}^{\infty} P_i = 1$ mit positiver Wahrscheinlichkeit, $\sum_{i=1}^{\infty} \mathbb{E}[P_i^2] < 1$ impliziert. Die zweite Behauptung des Satzes (\ref{first equation of Q theorem}) ist somit gezeigt. Für die letzte Behauptung \ref{third equation of Q theorem}, nutzen wir 
\[
    P(W=P_i|(P_i)_{i \in \mathbb{N}}) = P_i.  
\]  
und folgern
\begin{align}
\mathbb{E}[W] &= \mathbb{E}[\mathbb{E}[W|(P_i)_{i \in \mathbb{N}}]] \\
&= \mathbb{E}[\sum_{i=1}^{\infty}P_i\mathbb{P}(W = P_i|(P_i)_{i \in \mathbb{N}})] \\
&= \mathbb{E}[\sum_{i=1}^{\infty}P_i^2].
\end{align}
Nach Hoeffdings Gesetz der großen Zahlen (Satz \ref{hoeffding theorem}), gilt
\[
    \frac{\tilde{Y}_n}{n(n-1)} \rightarrow Q \hspace{5pt} \textnormal{P -f.s.} \hspace{5pt} \textnormal{für } n \to \infty.
\]
Es fehlt zu zeigen, dass $(\tilde{Y}_n-Y_n)/n^2 \rightarrow 0 $ P-f.s. für $n \to \infty$. \\
Sei 
\begin{equation}
H_n := max\{h(v) : v \in T_n\}
\end{equation}
die Höhe von $T_n := T_n^\mathcal{P}$ und
\begin{equation}
H^*_n := max\{f(\textbf{X}_k,\textbf{X}_l) : l < k \leq n\}.
\end{equation}
Wir definieren $v \prec w$, falls $v$ ein Vorfahren von $w$ ist. Dann gilt durch \ref{inequality for height}
\begin{align}
    0 \leq \tilde{Y}_n - Y_n &= 2\sum_{k=1}^{n}\sum_{v_l \prec v_k}(f(\textbf{X}_k,\textbf{X}_l)-h(v_k \wedge v_l)) \nonumber\\
    &\leq 2\sum_{k=1}^{n}\sum_{v_l \prec v_k}f(\textbf{X}_k,\textbf{X}_l) \nonumber \\
    &\leq 2\sum_{k=1}^{n}\sum_{v_l \prec v_k}H_n^* \nonumber \\
    &\leq 2\sum_{k=1}^{n}H_nH_n^*  = 2nH_nH_n^* \label{Abschätzung Tiefe}
\end{align}
wobei die letzte Ungleichung Wahr ist, da ein Knoten höchstens $H_n$ Vorfahren hat. \\
Nun hat der tiefst-liegende Knoten $k$ die Tiefe $H_n$ und falls der Vaterknoten mit $l$ bezeichnet wird, so gilt 
\begin{equation}
    H^*_n \geq f(\textbf{X}^k,\textbf{X}^l) \geq H_n - 1. \label{trivial inequality}
\end{equation}
Sei $m := m_n := \lceil c \log(n) \rceil$. Dann gilt, wie in der Gleichungskette \ref{Gleichungskette Tiefe},
\begin{align*}
\mathbb{P}(f(\textbf{X}^1,\textbf{X}^2)\geq m| \{\mathcal{P}^v, v \textnormal{ Knoten}\}) &=\sum_{i_1,...,i_m \in \mathbb{N}} \mathbb{P}( \{\bigcap_{j=1}^{m}\textbf{X}_{1,j}=\textbf{X}_{2,j} = i_j \}) \\ 
&=\sum_{i_1,...,i_m \in \mathbb{N}}(\prod_{j=1}^{m}P_{i_j}^{(i_1,...,i_{j-1})}) ^2.
\end{align*}
Und somit
\[
\mathbb{P}(f(\textbf{X}^1,\textbf{X}^2)\geq m) = \sum_{i_1,...,i_m \in \mathbb{N}}\prod_{j=1}^{m}\mathbb{E}[P_{i_j}^2]  =: a^m,
\]
für $a := \sum_{i=1}^{\infty}\mathbb{E}[P_i^2] < 1$. Wir können jetzt $H^*_n$ abschätzen. Dafür wählen wir $c \geq 4 / |\log{a}|$ und rechnen
\begin{align*}
\mathbb{P}(H^*_n \geq m) &\leq \sum_{l < k \leq n} \mathbb{P}(f(\textbf{X}_k,\textbf{X}_l) \geq m) = {n\choose 2}\mathbb{P}(f(\textbf{X}_k,\textbf{X}_l) \geq m)\\
&\leq n^2\mathbb{P}(f(\textbf{X}_k,\textbf{X}_l) \geq m) \leq n^2a^m \leq n^2a^{c \log{n}} \leq n^{-2}.
\end{align*}
Da $\sum_{n \geq 1}\frac{1}{n^2} < \infty$, folgt mit Borel-Cantelli $H_n \leq c \log{n}$ P-f.s. für $n \to \infty$. Nun gilt mit \ref{trivial inequality} und \ref{Abschätzung Tiefe}, dass $(\tilde{Y}_n - Y_n) \in O(n\log^2(n))$ und somit ist die Behauptung $(\tilde{Y}_n-Y_n)/n^2 \rightarrow 0 $ P-f.s. gezeigt. 
\end{proof}
\begin{Korollar}
    Für ein \textit{linear preferential attachment tree} mit Parametern $(\chi,\rho)$ mit $\chi+ \rho > 0$ existiert eine Zufallsvariable $Q$, sodass $Y(T_n^{\chi,\rho})/n^2 \to Q$ P-f.s. Weiter hat $Q$ den Erwartungswert 
    \[
        \mathbb{E}[Q] = \rho.
    \]
\end{Korollar}
\begin{proof}
    Das $Q$ existiert folgt direkt aus Satz \ref{big Q theorem}. Nach Satz \ref{Verteilung W}, ist $W \sim \beta(\rho,1)$ und hat somit Erwartungswert $\frac{\rho}{\rho + 1}$. Satz \ref{big Q theorem} liefert nun
    \begin{equation*}
        \mathbb{E}[Q] = \frac{\mathbb{E}[W]}{1-\mathbb{E}[W]} = \frac{\frac{\rho}{\rho + 1}}{\frac{1}{\rho+1}} = \rho
    \end{equation*}
\end{proof}