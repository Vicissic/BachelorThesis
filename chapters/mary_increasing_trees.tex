
\chapter{m-ary increasing trees}
% LTeX: language=de-DE
In diesem Kapitel werden die Gewichte der \textit{linear preferential attachment trees}, mit Parametern der Form $\chi < 0$ und $\rho > 0$ näher betrachtet. Wie schon im ersten Kapitel bemerkt, ändert die Normierung von allen Gewichten $w_k$ mit einer Konstante $c \in \mathbb{R}$ nicht die Verteilung der Bäume. Infolgedessen, betrachten wir $\chi = -1$ und prüfen, bei welchen $\rho \in \mathbb{R}^{+}$ es zu negative Gewichten kommt.\\
Für $\chi = -1$ sind die Gewichte der \textit{linear preferential trees} durch 
\[
    w_k = \chi k + \rho =  \rho - k 
\]
festgelegt. Für $k > \rho$ folgt daraus, dass $w_k < 0$. Daher ist die einzige Möglichkeit negative Gewichte zu vermeiden, $w_m = 0$, für ein $m \in \mathbb{N}$ sicherzustellen. In diesem Fall kann einem Knoten mit $m$ Kindern kein weiterer Knoten hinzugefügt werden, und die weiteren Gewichte $w_n, n > m$ sind von keiner Relevanz. Auf $\rho$ bezogen, heißt das, dass $\rho \in \mathbb{N}$, falls $\chi = -1$, bzw. $\rho= n\chi$, falls wir $\chi$ nicht zu $-1$ normieren.  
