\begin{section}{Verallgemeinerte CRP}
% LTeX: language=de-DE
Nachdem $n$ Gäste $k$ Tische besetzt haben, besetzt ein neuer Kunde einen Tisch nach folgenden Regeln:
\begin{enumerate}
    \item Der Gast besetzt ein Tisch $1 \leq i \leq k$ mit Wahrscheinlichkeit $\frac{n_i-\alpha}{n + \theta}$ 
    \item Der Gast besetzt einen neuen Tisch $k+1$ mit Wahrscheinlichkeit $\frac{\theta + k\alpha}{n + \theta}$.
\end{enumerate}
Damit diese tatsächlichen Wahrscheinlichkeiten entsprechen, müssen $\alpha$ und $\theta$ folgende Wahrscheinlichkeitsregeln erfüllen. \\
Das Haupttheorem ist nun folgendes.
\begin{theorem}
    Für jede der Parameter $(\alpha,\theta)$, welche die oben genannten Voraussetzungen erfüllen, generiert das CRP eine austauschbare  Partition von $\mathbb{N}$. Die dazu gehörende EPPF (siehe Def. 3) ist 
    \begin{equation}
   p_{\alpha,\theta}(n_1,...,n_k) = \frac{\displaystyle \prod_{i=0}^{k-2}(\theta + \alpha + i\alpha)\prod_{j=1}^{k}\prod_{r=1}^{n_i-1}(r-\alpha)}{\displaystyle\prod_{i=1}^{n-1}(\theta  + i)}.
    \end{equation}
    Die zugehörigen asymptotischen Frequenzen, in \textcolor{red}{size biased order of least elements} kann folgendermaßen repräsentiert werden.
    \[
    (\tilde{P_1},\tilde{P_2},\tilde{P_3}...) = (W_1,\bar{W_1}W_2,\bar{W_1}\bar{W_2}W_3...).
    \]
    Wobei $W_i$ unabhängig voneinander, nach $\beta(1 - \alpha, \theta + i\alpha)$ verteilt sind und $\bar{W_i}:= 1 - W_i$.
\end{theorem}  
\begin{proof}
   Wir beginnen mit der Austauschbarkeit. Dafür prüfen wir zuerst, dass $p_{\alpha,\theta}$ tatsächlich unserer EPPF entspricht. Wir nutzen \ref{Austauschbar Lemma}.
\end{proof}

\end{section}