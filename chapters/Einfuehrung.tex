\chapter*{Einführung}
% LTeX: language=de-DE
Die vorliegende Bachelorarbeit ist eine Ausarbeitung von der wissenschaftlichen Publikation \glqq Random recursive Trees and Preferential Attachment Trees are random split trees\grqq von Svante Janson \cite{janson2019random}.\\
Die Publikation von Janson verallgemeinert die von Devroye \cite{devroye1998universal} eingeführten \glqq Split Bäume\grqq, indem auch Bäume mit unendlichen Ausgangsgraden, also Anzahl von Kindern pro Wurzel, auch zugelassen werden. Konkret, werden sogenannter \glqq Linear Preferential Attachment Trees\grqq in die Vorlage der Split-Bäume aufgenommen. Je nach Parameter Wahl umfassen die \textit{Preferential Trees} unter anderem die \textit{m-ary, standard preferential und zufälligen rekursiven} Bäume und Eigenschaften der jeweiligen Bäume werden in \textcolor{red}{Füge später hinzu} näher betrachtet. Da die Publikation stark auf \textit{Kingman's theory of exchangeable partitions} \cite{kingman1978representation,kingman1982coalescent} als auch auf \textit{Pitman's chinese restaurant process} \cite{pitman2006combinatorial,pitman1995exchangeable} beruht, wird zuerst ein Überblick über diese Themengebiete verschaffen.