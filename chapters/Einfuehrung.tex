\chapter{Einführung}
% LTeX: language=de-DE
Die vorliegende Bachelorarbeit ist eine Ausarbeitung von der wissenschaftlichen Publikation \glqq Random Recursive Trees and Preferential Attachment Trees are Random Split Trees\grqq, von Svante Janson \cite{janson2019random}.\\
Die Publikation von Janson verallgemeinert die von Devroye \cite{devroye1998universal} eingeführten \textit{Split trees}, indem auch Bäume mit unendlichen Ausgangsgraden, also Anzahl von Kindern pro Wurzel, zugelassen werden. Präziser formuliert, werden sogenannte \textit{Linear Preferential Attachment Trees} in die Vorlage der Split-Bäume aufgenommen. Je nach Parameter Wahl umfassen die \textit{Preferential Trees} unter anderem die \textit{m-ary, standard preferential und zufälligen rekursiven} Bäume und Eigenschaften der jeweiligen Bäume werden in \textcolor{red}{Füge später hinzu} näher betrachtet. Da die Publikation stark auf \textit{Kingman's theory of exchangeable partitions} \cite{kingman1978representation,kingman1982coalescent} als auch auf \textit{Pitman's chinese restaurant process} \cite{pitman2006combinatorial,pitman1995exchangeable} beruht, wird im ersten Kapitel ein Überblick über diese Themengebiete verschaffen. Anschließend wird das Haupttheorem zur Einbettung in Kapitel 2 bewiesen, gefolgt von einer Anwendung zu Tiefen in Kapitel 3. Im letzten Kapitel wird ein Spezialfall der \textit{Preferential Trees} gesondert behandelt.
