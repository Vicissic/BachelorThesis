\chapter*{Anhang}
% LTeX: language=de-DE
\begin{lemma}
\label{combinatorial lemma}
Die Anzahl der Partitionen von $[n]$ mit Klassengröße in Reihenfolge ihres Erscheinens ist durch \ref{combinatorial equation} gegeben.
\end{lemma}
\begin{proof}
    Das $\#(\textbf{n})$ tatsächlich die Form \ref{combinatorial equation} hat, lässt sich durch Betrachtung von
    \begin{align*}
   \Omega_{\textbf{n}} = \{ \textbf{x}:= (x_{n_1}^{1},x_{n_1}^{2},...,x_{n_1}^{n_1},x_{n_2}^{1},x_{n_2}^{2},...,x_{n_k}^{n_k})|& \textbf{n}:= (n_1,...,n_k), \sum_{i=1}^{k}n_i=  n \\
   &,\textbf{x} \in \mathcal{P}(n)\\
   &, x_{n_i}^j < x_{n_i}^{j+1}\hspace{5pt} \forall  i \in \{1...k\} \forall j \in \{1...n_i-1\} \\
   &, x^1_{n_i} < x^1_{n_i+1} \hspace{5pt} \forall i \in \{1...k\}\}
    \end{align*} 
    nachweisen. Es gilt
    \[
   \#(\textbf{n}) =  |\Omega_\textbf{n}|
    \]
    und wir berechnen nun $|\Omega_\textbf{n}|$. Es gibt $n!$ Möglichkeiten $[n]$ zu permutieren, nach den Bedingungen von $\Omega_n$ muss aber $x_{n_i}^1$ für alle $i$ kleiner sein als jede darauffolgende Zahl in der Liste. Formal beschrieben bedeutet das, falls $\textbf{x}_i, 1 \leq i \leq n$ die i-te Komponente von $\textbf{x}$ ist, so ist $\textbf{x}_{n_i} < \textbf{x}_{j}$ $\forall i \in \{n_i+1,...,n\}$. Sei  
    \begin{align*}
   \Omega^1_{\textbf{n}} = \{ \textbf{x}:= (x_{n_1}^{1},x_{n_1}^{2},...,x_{n_1}^{n_1},x_{n_2}^{1},x_{n_2}^{2},...,x_{n_k}^{n_k})|& \textbf{n}:= (n_1,...,n_k), \sum_{i=1}^{k}n_i=  n \\
   &,\textbf{x} \in \mathcal{P}(n)\\
   &, \textbf{x}_{n_i} < \textbf{x}_j \forall i \in \{1...k\},\forall j \in \{n_i+1,...,n\}\}
    \end{align*} 
    so ist
    \[
    |\Omega^1_{\textbf{n}}| = \frac{n!}{n(n-n_1)(n-n_2-n_1)...(n-\sum_{i=1}^{k-1}n_i)}.
    \]
    Um von $|\Omega_\textbf{n}^1|$ auf $|\Omega_\textbf{n}|$ zu kommen fehlt uns in jedem der einzelnen Komponenten $n_i$ die Permutationen der $n_i -1$ nachfolgenden Zahlen nach jedem der $\textbf{x}_{n_i}$ zu zählen. Da es aber $\prod_{i=1}^{k}(n_i-1)!$ davon gibt, gilt 
    \[
     |\Omega_\textbf{n}| = \frac{|\Omega_\textbf{n}^1|}{\prod_{i=1}^{k}(n_i-1)!}
    \]
    und die Behauptung ist gezeigt.
\end{proof}
\begin{lemma}
    \label{lemma cases crp}
    Damit ein Paar $(\alpha,\theta)$ einen wohldefinierten \textit{chinese restaurant process} definiert, muss $(\alpha,\theta)$ eine von folgenden beiden Bedingungen erfüllen:
    \begin{enumerate}
        \item $0 \leq \alpha < 1$ und $\theta > -\alpha$
        \item $\alpha < 0$ und $\theta = -m \alpha$ für ein $m \in \mathbb{N}$.
    \end{enumerate} 
\end{lemma}
\begin{proof}
Es muss geprüft werden, ob $\frac{n_j- \alpha}{n+\theta}$ und $\frac{\theta + k\alpha}{n + \theta}$  stets zwischen $0$ und $1$ liegen. Wir betrachten jetzt 5 Beispielfälle einzeln, die restlichen folgen mit analogen Argumenten.\\
\\
\textit{Fall 1}: $\alpha =1$\\
Dies entspricht dem Fall, dass jeder Gast an einen neuen Tisch gesetzt wird. Falls wir die Beta Verteilung um $\beta(0,l) = \delta_0, l \in \mathbb{R}$, also die deterministische Zufallsvariable mit Wert 0 erweitern, so gilt Satz \ref{Main theorem CRP} auch für diesen Fall. Die dazugehörige Folge $(P_1,P_2...)$ wäre dann die $0$ Folge.
\\
\textit{Fall 2}: $\alpha >1$ und $-1<\theta< 0$\\
Bei $n=1$ ist $k=1$ und somit 
\[\frac{\theta + k\alpha}{n + \theta} = \frac{\theta + \alpha}{1 + \theta} > 1
\] \\ 
\textit{Fall 3}: $\alpha >1$ und $\theta \neq -l\alpha, l \in \mathbb{N}$\\
In diesem Fall wollen wir zeigen, dass 
\[
\frac{n_j-\alpha}{n + \theta}<0
\]
 für mögliche $n_j \in \mathbb{N}$. Es reicht, falls wir zeigen, dass für alle $n \in \mathbb{N}$ wir eine Anordnung von Tischen finden, sodass mindestens ein Tisch mit nur einem Gast besetzt ist. Dann würden $n \in \mathbb{N}$ existieren, nämlich $\forall n > \lceil l \alpha \rceil$, sodass 
 \[
 \frac{n_j-\alpha}{n + \theta}= \frac{1-\alpha}{n + \theta}<0.
 \] 
 Nun ist für jede mögliche Anordnung von Tischen mit $n$ Gästen, die Wahrscheinlichkeit einen neuen Tisch zu beschaffen $\frac{\theta + k\alpha}{n + \theta} \neq 0$ und somit wird es im nächsten Schritt ein Tisch mit nur einem Gast geben. Dieser Fall ist deshalb ausgeschlossen.\\
\textit{Fall 4}: $\alpha >1$ und $\theta = -\alpha$\\
Dieser Fall ist Analog zu dem Fall $\alpha < 0$ und $-m \alpha$ und Satz \ref{Main theorem CRP} kann auch mit diesem Fall erweitert werden.\\
\textit{Fall 5}: $\alpha >1$ und $\theta = -l\alpha, l \geq 2, l \in \mathbb{N}$\\
Wir schließen zuerst den Fall $-\theta \in \mathbb{N}$ aus, in diesem Fall würde es für ein $n \in \mathbb{N}$ zu dem Fall kommen, wo der Nenner gleich $0$ und somit undefiniert wäre. Infolgedessen, ist auch $\alpha \notin \mathbb{N}$, denn wenn $\alpha \in \mathbb{N}$, dann auch $-\theta \in \mathbb{N}$ wäre.\\
Wir betrachten nun die Anordnung wo jeder Gast sich an den ersten Tisch setzt. Die Wahrscheinlichkeit, dass sich der $(n+1)$-te Gast sich an den ersten Tisch setzt, da $n_j = n$, ist 
\[
    \frac{n- \alpha}{n - l \alpha}.   
\]
Nach der ersten Bemerkung, dass $\alpha \notin \mathbb{N}$, ist diese Wahrscheinlichkeit nie $0$, also muss 
\[
   0<  \frac{n- \alpha}{n - l \alpha} \leq 1.   
\]
Da aber es eine natürliche Zahl $n$ geben muss, die zwischen $\alpha$ und $l \alpha$ liegt, ist die Wahrscheinlichkeit für diese Zahl $<0$ und somit ist diese Möglichkeit auch nicht zulässig.
\\
\\
Die restlichen Fälle können mit analogen Argumenten geprüft werden.
\end{proof}