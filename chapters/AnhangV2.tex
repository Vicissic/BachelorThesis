\chapter*{Anhang}
% LTeX: language=de-DE
\begin{lemma}
Die Anzahl der Partitionen von $[n]$ mit Klassengröße in Reihenfolge ihres Erscheinens ist durch \ref{combinatorial equation} gegeben.
\end{lemma}
\begin{proof}
    Das $\#(\textbf{n})$ tatsächlich die Form \ref{combinatorial equation} hat, lässt sich durch Betrachtung von
    \begin{align*}
   \Omega_{\textbf{n}} = \{ \textbf{x}:= (x_{n_1}^{1},x_{n_1}^{2},...,x_{n_1}^{n_1},x_{n_2}^{1},x_{n_2}^{2},...,x_{n_k}^{n_k})|& \textbf{n}:= (n_1,...,n_k), \sum_{i=1}^{k}n_i=  n \\
   &,\textbf{x} \in \mathcal{P}(n)\\
   &, x_{n_i}^j < x_{n_i}^{j+1}\hspace{5pt} \forall  i \in \{1...k\} \forall j \in \{1...n_i-1\} \\
   &, x^1_{n_i} < x^1_{n_i+1} \hspace{5pt} \forall i \in \{1...k\}\}
    \end{align*} 
    nachweisen. Es gilt
    \[
   \#(\textbf{n}) =  |\Omega_\textbf{n}|
    \]
    und wir berechnen nun $|\Omega_\textbf{n}|$. Es gibt $n!$ Möglichkeiten $[n]$ zu permutieren, nach den Bedingungen von $\Omega_n$ muss aber $x_{n_i}^1$ für alle $i$ kleiner sein als jede darauffolgende Zahl in der Liste. Formal beschrieben bedeutet das, falls $\textbf{x}_i, 1 \leq i \leq n$ die i-te Komponente von $\textbf{x}$ ist, so ist $\textbf{x}_{n_i} < \textbf{x}_{j}$ $\forall i \in \{n_i+1,...,n\}$. Sei  
    \begin{align*}
   \Omega^1_{\textbf{n}} = \{ \textbf{x}:= (x_{n_1}^{1},x_{n_1}^{2},...,x_{n_1}^{n_1},x_{n_2}^{1},x_{n_2}^{2},...,x_{n_k}^{n_k})|& \textbf{n}:= (n_1,...,n_k), \sum_{i=1}^{k}n_i=  n \\
   &,\textbf{x} \in \mathcal{P}(n)\\
   &, \textbf{x}_{n_i} < \textbf{x}_j \forall i \in \{1...k\},\forall j \in \{n_i+1,...,n\}\}
    \end{align*} 
    so ist
    \[
    |\Omega^1_{\textbf{n}}| = \frac{n!}{n(n-n_1)(n-n_2-n_1)...(n-\sum_{i=1}^{k-1}n_i)}.
    \]
    Um von $|\Omega_\textbf{n}^1|$ auf $|\Omega_\textbf{n}|$ zu kommen fehlt uns in jedem der einzelnen Komponenten $n_i$ die Permutationen der $n_i -1$ nachfolgenden Zahlen nach jedem der $\textbf{x}_{n_i}$ zu zählen. Da es aber $\prod_{i=1}^{k}(n_i-1)!$ davon gibt, gilt 
    \[
     |\Omega_\textbf{n}| = \frac{|\Omega_\textbf{n}^1|}{\prod_{i=1}^{k}(n_i-1)!}
    \]
    und die Behauptung ist gezeigt.
\end{proof}