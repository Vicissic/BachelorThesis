\chapter*{Chapter 1} 
% LTeX: language=de-DE
\begin{theorem}
\cite[Lemma 3.1]{janson2019random}
   Mit den linearen Gewichten $w_k = \chi k + p $, hat ein Baum $T$ mit $m$ Knoten ein Gesamtgewicht $w(T) = (m-1)\chi + m\rho = m(\chi + \rho) - \chi$.
\end{theorem}
\begin{proof}
    Seien $d_1...d_m$ die Ausgangsgrade der jeweiligen Knoten im Baum. Dann ist die Summe der Ausgangsgrade, $\sum_{i=1}^{m}d_i = m-1$ und somit:
    \[
     w(T) = \sum_{i=1}^{m}w_i = \sum_{i=1}^{m} (\chi d_i + \rho) = m \rho + \chi \sum_{i = 1}^{m} d_i = m(\chi + \rho) - \chi  
    \]
\end{proof}
\begin{theorem} \cite[Theorem 3.2]{janson2019random}
    Sei $(T_n)_1^\infty =(T_n^{\chi,\rho})_1^\infty = $ eine Folge von \linpreft, mit den Kindern von der Wurzel in der Reihenfolge ihres Erscheinens beschriftet. Sei $N_j(n) := |T^j_n|$, die Größe des j-ten \PsubT von $T_n$. Dann gilt 
    \[ 
        \forall j \geq 1  \hspace{6pt} \frac{N_j(n)}{n}  \rightarrow P_j \hspace{6pt} \text{für} \hspace{6pt} n \rightarrow \infty \hspace{6pt} \text{P-f.s.}.
  \]
Wobei die Zufallsvariablen  $P_j$ Verteilung $GEM(\frac{\chi}{\chi+\rho},{\frac{\rho}{\chi+\rho}})$ haben. \textcolor{red}{add case 0} 
\end{theorem}
\begin{proof}
    Da wir im Allgemeinen $\chi + \rho > 0 $ annehmen dürfen, und da die Multiplikation von allen $w_k$ mit einer Konstante nicht den Baum ändert, können wir o.B.d.A   $\chi + \rho = 1 $ annehmen. Das Argument folgt nun durch Übersetzung von \PsubT zu Zyklen in Permutationen und eine Bijektion zwischen \linpreft und Permutationen. Wir beginnen mit der Bijektion. \\
    Dafür beschriften wir die Knoten sukzessiv, beginnend mit der Wurzel als 0, nach der Reihenfolge vom Hinzufügen zum Baum (der $i$-ter Knoten der hinzugefügt wird, wird mit $i-1$ beschriftet). Die Wurzel, die mit 0 beschriftet wird, wird identifiziert mit der leeren Permutation und der Zweite Knoten, welcher mit 1 beschriftet wird, wird mit dem Zyklus $(1)$ identifiziert. Wir identifizieren nun unsere Bäume mit Permutationen induktiv, nach folgenden zwei Regeln:
    \begin{enumerate}
        \item Falls der Knoten mit Beschriftung $i$ an die Wurzel angehängt wird, so fügen wir zur bestehenden Zyklus-Dekomposition, die Fixpunkt-Permutation $(i)$ hinzu.
        \item Falls der Knoten mit Beschriftung $i$ an ein anderen Knoten mit Beschriftung $j$, $j \neq 0$ angehängt wird, so fügen wir in den Zyklus $C$, welcher $j$ enthält, $i$ direkt nach $j$ hinzu. Bedeutet also, falls im ursprünglichen Zyklus $j \rightarrow k$ abgebildet wurde, im nächsten Schritt $j \rightarrow i \rightarrow k$ abgebildet wird.
    \end{enumerate}
Es ist einfach zu zeigen, dass diese Abbildung von Bäumen zu Permutation, welche durch diese Konstruktion entsteht, bijektiv ist und somit können wir die Theorie die für 
\end{proof}
